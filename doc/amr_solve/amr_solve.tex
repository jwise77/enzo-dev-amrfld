\documentclass[letterpaper,10pt]{article}
\usepackage{geometry}   % See geometry.pdf to learn the layout
                        % options.  There are lots.
\usepackage[latin1]{inputenc}
\usepackage{graphicx}
\usepackage{epstopdf}
\usepackage{amsmath,amsfonts,amssymb}

\DeclareGraphicsRule{.tif}{png}{.png}{`convert #1 `dirname #1`/`basename #1 .tif`.png}

\renewcommand{\(}{\left(}
\renewcommand{\)}{\right)}
\newcommand{\vb}{{\bf v}_b}
\newcommand{\xvec}{{\bf x}}
\newcommand{\Omegabar}{\bar{\Omega}}
\newcommand{\rhob}{\rho_b}
\newcommand{\dt}{\Delta t}
\newcommand{\Eot}{E^{OT}}
\newcommand{\Ef}{E_f}
\newcommand{\sighat}{\hat{\sigma}}
\newcommand{\Fnu}{{\bf F}_{\nu}}
\newcommand{\Pnu}{\overline{\bf P}_{\nu}}
\newcommand{\R}{I\!\!R}
\newcommand{\Rthree}{\R^3}
\newcommand{\eh}{e_h}
\newcommand{\ec}{e_c}
\newcommand{\Edd}{\mathcal F}
\newcommand{\Eddnu}{\Edd_{\nu}}
\newcommand{\mn}{{\tt n}}
\newcommand{\mB}{\mathcal B}
\newcommand{\mC}{{\mathcal C}}
\newcommand{\mL}{{\mathcal L}}
\newcommand{\mD}{{\mathcal D}}
\newcommand{\mDnu}{\mD_{\nu}}
\newcommand{\mCnu}{\mC_{\nu}}
\newcommand{\mLnu}{{\mathcal L}_{\nu}}
\newcommand{\mCe}{\mC_e}
\newcommand{\mLe}{\mL_e}
\newcommand{\mCn}{\mC_{\mn}}
\newcommand{\mLn}{\mL_{\mn}}
\newcommand{\amrsolve}{{\tt AMRSolve} }
\newcommand{\enzo}{{\tt Enzo} }
\newcommand{\hypre}{{\tt HYPRE} }
\newcommand{\cpp}{{\tt C++} }


\textheight 9truein
\textwidth 6.5truein
\addtolength{\oddsidemargin}{-0.25in}
\addtolength{\evensidemargin}{-0.25in}
\addtolength{\topmargin}{-0.5in}
\setlength{\parindent}{0em}
\setlength{\parskip}{2ex}


\author{Daniel R. Reynolds and James O. Bordner}
\title{\amrsolve User Guide}


\begin{document}
\maketitle

\section{Introduction}
\label{sec:intro}

The \amrsolve package is an \enzo add-on that strives to achieve two
goals: (a) provide a smoother approach for iteration over grids in an
\enzo AMR hierarchy, both within and between levels, and within and
between MPI tasks, and (b) to provide a straightforward interface for
defining linear systems on block-structured adaptive mesh hierarchies
and solving them using iterative linear solvers provided by the \hypre
solver library.  The first of these alone allows \amrsolve to
significantly benefit \enzo development of new physics modules that
require inter-grid communication, if only due to the ease of
development.  However, the combination of the two goals of \amrsolve
has allowed us to construct two \enzo solvers for implicit systems
defined on AMR grids: self-gravity and flux-limited-diffusion
radiation transport.

In this user guide, we first provide a brief overview of the \amrsolve
package structure, including a description of its suite of \cpp
classes and iterators.  We then focus on the self-gravity solver, 
{\tt AMRGravitySolve}. We will conclude with a discussion of the FLD
radiation transport solver, {\tt AMRFLDSplit}. 


\section{\amrsolve}
\label{sec:amrsolve}

The \amrsolve package contains a relatively large number of classes,
providing a range of utilities from description of the \enzo scalar
type to MPI communicator wrappers, to HYPRE solver interfaces.  All
\amrsolve package routines are held in the \enzo directory 
{\tt src/amr\_solve/}, which are symbolically linked during \enzo
compilation to the {\tt src/enzo/} directory.  We discuss each of
these \cpp classes below, focusing more on those utilities that
interface most closely with \enzo.
\begin{itemize}
\item {\tt AMRsolve\_Domain} -- small class containing the problem
  dimensionality and root-grid extents.
\item {\tt AMRsolve\_Hierarchy} -- this class represents {\tt Enzo}'s
  AMR hierarchy.  However, unlike the \enzo hierarchy structures, this
  class comes with friend classes that provide an invaluable set of
  iterators throughout the hierarchy:
  \begin{itemize}
  \item {\tt ItHierarchyGridsLocal} -- iterator class for visiting all
    task-local grids within an \\
    {\tt AMRsolve\_Hierarchy} object.
  \item {\tt ItHierarchyGridsAll} -- iterator class for visiting all
    grids within an {\tt AMRsolve\_Hierarchy} object (akin to the
    linked-list iteration type through the \enzo hierarchy).
  \item {\tt ItHierarchyLevels} -- iterator class for visiting all
    levels in an an {\tt AMRsolve\_Hierarchy} object, running from
    coarsest to finest.
  \end{itemize}
  An {\tt AMRsolve\_Hierarchy} object is set up within \enzo by
  ``attaching'' it to an \enzo {\tt LevelArray}.  There are currently
  two such routines, one for gravity problems 
  ({\tt enzo\_attach\_grav}) and the other for radiation problems 
  ({\tt enzo\_attach\_fld}).  Upon solution of the necessary
  gravity/radiation problem, the {\tt AMRsolve\_Hierarchy} object may
  be detached from enzo via the {\tt enzo\_detach} routine, allowing
  for adaptivity of the mesh before re-attachment.
\item {\tt AMRsolve\_Level} -- small class for encapsulating a single
  level of an AMR hierarchy.  Again, the true utility in this class is
  it's friend class iterators:
  \begin{itemize}
  \item {\tt ItLevelGridsLocal} -- iterator class for visiting all
    task-local grids in an {\tt AMRsolve\_Level} object.
  \item {\tt ItLevelGridsAll} -- iterator class for visiting all
    grids in an {\tt AMRsolve\_Level} object.
  \end{itemize}
\item {\tt AMRsolve\_Grid} -- class for representing a grid patch in
  an AMR hierarchy.  Each grid is an orthogonal grid of zones in d
  dimensions.  These grids contain a small amount of data for
  interfacing with \hypre solvers, as well as descriptors of the
  corresponding \enzo grid object.  However, these grids do store
  pointers to some of the corresponding \enzo data fields that are
  used in solving up gravity and radiation problems.  The true benefit
  of this class is not to merely reproduce a descriptor of each \enzo
  grid object; instead it comes from the suite of associated
  iterators, that obviate the need to iterate over a full hierarchy to
  determine inter-grid relationships:
  \begin{itemize}
  \item {\tt id\_parent\_} -- this member points to the parent {\tt
    AMRsolve\_Grid}.
  \item {\tt ItGridNeighbors} -- this class iterates over all
    neighboring grids of an {\tt AMRsolve\_Grid}.
  \item {\tt ItGridChildren} -- this class iterates over all child
    grids of an {\tt AMRsolve\_Grid}.
  \end{itemize}
\item {\tt AMRsolve\_Faces} -- each {\tt AMRsolve\_Grid} object has a
  corresponding {\tt AMRsolve\_Faces} object, that identifies what the
  boundaries of the grid are adjacent to (e.g.~coarser cells, finer
  cells, external boundary).
\item {\tt AMRsolve\_Mpi} -- small wrapper class containing the MPI
  communicator, communicator size, and MPI task index.
\item {\tt AMRsolve\_Parameters} -- small class for storing and
  accessing run-time parameters using key-value pairs.
\item {\tt AMRsolve\_Point} -- small class for representing point
  masses (used in some self-gravity tests).
\item {\tt AMRsolve\_Problem} -- small class for setting up
  self-gravity test problems.
\item {\tt AMRsolve\_scalar.h} -- file to define a 'Scalar' type (as
  opposed to re-defining 'float').
\item {\tt AMRsolve\_Hypre\_Grav} -- class for interfacing between the
  \enzo routine {\tt AMRGravitySolve} and \hypre, to perform scalable,
  self-consistent self-gravity solves over a full AMR hierarchy.  This
  will be discussed in further detail in Section
  \ref{sec:AMRGravitySolve}.
\item {\tt AMRsolve\_Hypre\_FLD} -- class for interfacing between the
  \enzo class {\tt AMRFLDSplit} and \hypre, to perform scalable,
  self-consistent flux-limited-diffusion radiation transport solves
  over a full AMR hierarchy.  This will be discussed in further detail
  in Section \ref{sec:AMRFLDSplit}.
\item {\tt AMRsolve\_HG\_Prec} -- class to provide a
  ``hierarchical-grid'' preconditioner for fully implicit linear
  systems posed on block-structured AMR meshes.  This preconditioner
  is designed to be plugged into a \hypre iterative linear solver,
  such as BiCGStab or GMRES, and aids in constructing a scalable
  solver for elliptic (gravity) or parabolic (FLD) linear systems.
\end{itemize}



\section{The {\tt AMRGravitySolve} Routine}
\label{sec:AMRGravitySolve}


\section{The {\tt AMRFLDSplit} Class}
\label{sec:AMRFLDSplit}








\subsection{Flux-limited diffusion radiation model}
\label{sec:rad_model}

We begin with the equation for flux-limited diffusion radiative
transfer in a cosmological medium \cite{ReynoldsHayesPaschosNorman2009},
\begin{equation}
\label{eq:radiation_PDE}
  \partial_{t} E + \frac1a \nabla\cdot\(E\vb\) =
    \nabla\cdot\(D\,\nabla E\) - \frac{\dot{a}}{a} E - c\kappa E + \eta,
\end{equation}
where here the comoving radiation energy density $E$, emissivity
$\eta$ and opacity $\kappa$ are functions of space and time.  In this
equation, the frequency-dependence of the radiation energy has been
integrated away, under the premise of an assumed radiation energy
spectrum, 
\begin{align}
  \notag
  & E_{\nu}(\nu,\xvec,t) = \tilde{E}(\xvec,t) \chi(\nu), \\
  \notag
  \Rightarrow & \\
  \label{eq:spectrum}
  & E(\xvec,t) = \int_{\nu_{HI}}^{\infty} E_{\nu}(\nu,\xvec,t)\,\mathrm{d}\nu 
    = \tilde{E}(\xvec,t) \int_{\nu_{HI}}^{\infty} \chi(\nu)\,\mathrm{d}\nu,
\end{align}
where $\tilde{E}$ is an intermediate quantity (for analysis) that is
never computed.  We note that if the assumed spectrum is the Dirac
delta function, $\chi(\nu) = \delta_{\nu_{HI}}(\nu)$, $E$ is a
monochromatic radiation energy density at the ionization threshold of
HI, and the $-\frac{\dot{a}}{a}E$ term (obtained through integration
by parts of the redshift term
$\frac{\dot{a}}{a}\partial_{\nu}E_{\nu}$) is omitted from
\eqref{eq:radiation_PDE}. Similarly, the emissivity function
$\eta(\xvec,t)$ relates to the true emissivity 
$\eta_{\nu}(\nu,\xvec,t)$ by the formula
\begin{equation}
\label{eq:emissivity}
  \eta(\xvec,t) = \int_{\nu_{HI}}^{\infty}\eta_{\nu}(\nu,\xvec,t)\,\mathrm{d}\nu.
\end{equation}

The function $D$ in the above equation \eqref{eq:radiation_PDE} is
the {\em flux-limiter} that depends on $E$, $\nabla E$ and the 
opacity $\kappa$,  
\[
   D(E) = \text{diag}\( D_1(E),\, D_2(E),\, D_3(E) \),
\]
where the directional limiters $D_i(E)$ are given by \cite{Morel2000}
\begin{align}
  \label{eq:Larsen_limiter}
   D_i(E) = \frac{c}{\sqrt{(3\kappa)^2 + R_i^2}}, \qquad
   R_i(E) = \max\left\{\frac{|\partial_i E|}{E}, 10^{-20} \right\}.
\end{align}




\section{LTE couplings}
\label{sec:lte_model}

For problems in which chemical ionization is unimportant, we may
assume that the gas is in local thermodynamic equilibrium.  In this
case we do not need to couple the radiation energy to a model for
chemical ionization, we therefore couple the radiation to the specific
gas energy equation, 
\begin{align}
  \label{eq:cons_energy}
  \partial_t e + \frac1a\vb\cdot\nabla e &=
    - \frac{2\dot{a}}{a}e
    - \frac{1}{a\rhob}\nabla\cdot\left(p\vb\right) 
    - \frac1a\vb\cdot\nabla\phi + G - \Lambda.
\end{align}
All but the final two terms in \eqref{eq:cons_energy} are already
handled by Enzo's existing hydrodynamics solver infrastructure.  In
this module, we therefore consider a specific energy correction equation 
\begin{align}
  \label{eq:cons_energy_correction}
  \partial_t e_c &= -\frac{2\dot{a}}{a}e_c + G - \Lambda,
\end{align}
that we use to correct Enzo's original gas energy to include radiation
couplings.  Here $G$ is the local heating rate,
\begin{align}
\label{eq:G_LTE}
  G &= \frac{c \kappa}{\rhob} E,
\end{align}
and $\Lambda$ corresponds to the local cooling rate,
\begin{align}
\label{eq:Lambda_LTE}
  \Lambda = \frac{\eta}{\rhob} .
\end{align}
The user-defined energy mean opacity $\kappa$ is
spatially-homogeneous, and is given by the formula
\begin{align}
\label{eq:opacityE}
  \kappa = C_0 \left(\frac{\rhob}{C_1}\right)^{C_2}
\end{align}
(the constants $C_0\to C_2$ are input by the user), and $\eta$ is
a black-body emissivity given by 
\begin{align}
\label{eq:etaBB}
  \eta = 4\kappa\,\sigma_{SB}\,T^4,
\end{align}
where $\sigma_{SB}$ is the Stefan-Boltzmann constant [$5.6704\times
10^{-5}$ erg s$^{-1}$ cm$^{-2}$ K$^{-4}$], and $T$ is the gas
temperature [K]. 



\section{Chemistry-dependent couplings}
\label{sec:chem_model}

In general, radiation calculations in Enzo are used in simulations
where chemical ionization states are important.  For these situations, 
we couple the radiation equation \eqref{eq:radiation_PDE} with
equations for both the specific gas energy correction and the
ionization dynamics of Hydrogen,
\begin{align}
  \notag
  \partial_t e_c &= -\frac{2\dot{a}}{a}e_c + G - \Lambda, \\
  \label{eq:hydrogen_ionization}
  \partial_t \mn_{HI} + \frac{1}{a}\nabla\cdot\(\mn_{HI}\vb\) &=
    \alpha^{rec} \mn_e \mn_{HII} - \mn_{HI} \Gamma_{HI}^{ph}. 
\end{align}
Here, $\mn_{HI}$ is the comoving Hydrogen I number density.  The
recombination rate $\alpha^{rec}$ is given by the case-B
recombination rate, 
\begin{equation}
\label{eq:alphaB}
\alpha^{rec} = 2.753\times 10^{-14} \left(\frac{3.15614\times 10^5}{T}\right)^{3/2} 
                   \left(1+\left(\frac{3.15614\times 10^5}{2.74\, T}\right)^{0.407}\right)^{-2.242}.
\end{equation}

In this model, the gas heating and cooling rates are
chemistry-dependent, 
\begin{align}
  \label{eq:G_nLTE}
  G &= \frac{c\,E\,\mn_{HI}}{\rhob} 
    \left[\int_{\nu_{HI}}^{\infty} \sigma_{HI}\, \chi_E
    \left(1-\frac{\nu_{HI}}{\nu}\right)\, d\nu\right] \bigg/
    \left[\int_{\nu_{HI}}^{\infty} \chi_E d\nu\right], \\
\label{eq:Lambda_nLTE}
  \Lambda &= \frac{\mn_e}{\rhob}\bigg[\text{ce}_{HI}\, \mn_{HI} 
  + \text{ci}_{HI}\, \mn_{HI} + \text{re}_{HII}\, \mn_{HII} + \text{brem}\,
  \mn_{HII} \\
  \notag &\qquad+ \frac{m_h}{\rho_{units}\, a^3} \left(\text{comp}_1\, (T-\text{comp}_2) 
    + \text{comp}_{X}\, (T-\text{comp}_{T})\right) \bigg].
\end{align}
The temperature-dependent cooling rates
$\text{ce}_{HI}$, $\text{ci}_{HI}$, $\text{re}_{HII}$, $\text{brem}$,
$\text{comp}_1$, $\text{comp}_2$, $\text{comp}_{X}$ and
$\text{comp}_{T}$ are all taken from Enzo's built-in rate tables.

Moreover, the frequency-integrated opacity is now chemistry-dependent,
\begin{equation}
\label{eq:opacityHI}
  \kappa \ = \ 
  \left[\int_{\nu_{HI}}^{\infty} \kappa_{\nu}\,E_{\nu}\,d\nu\right] \bigg/
  \left[\int_{\nu_{HI}}^{\infty} E_{\nu}\,d\nu\right] \ = \ 
  \left[\mn_{HI} \int_{\nu_{HI}}^{\infty}
    \chi_E\,\sigma_{HI}\,d\nu\right] \bigg/
  \left[\int_{\nu_{HI}}^{\infty} \chi_E\,d\nu\right],
\end{equation}
where these integrals with the assumed radiation spectrum $\chi(\nu)$
handle the change from the original frequency-dependent radiation
equation to the integrated grey radiation equation.

\subsection{Interface to external Enzo chemistry/cooling solvers}
\label{sec:chem_model_external}

If {\tt gFLDSplit} is not used to evolve equations
\eqref{eq:cons_energy_correction} or \eqref{eq:hydrogen_ionization},
then {\tt gFLDSplit} will instead fill in the spatially-dependent
coupling terms $G$, $\Gamma_{HI}^{ph}$ and possibly
$\Gamma_{HeI}^{ph}$ and $\Gamma_{HeII}^{ph}$, for use by external
ionization/heating solvers.  These are held in the baryon fields 
{\tt PhotoGamma}, {\tt kphHI}, {\tt kphHeI} and {\tt kphHeII},
respectively.  We note that since the radiation fields evolved by 
{\tt gFLDSplit} do not include frequencies below $\nu_{HI}$, the
baryon field {\tt kdissH2I} is always set to 0.




\section{Numerical solution approach}
\label{sec:solution_approach}

We solve these models in an operator-split fashion, where we solve
the radiation equation \eqref{eq:radiation_PDE} separately from the
more tightly-coupled gas energy correction and chemistry equations
\eqref{eq:cons_energy_correction} and \eqref{eq:hydrogen_ionization},
which are evolved together.  These solves are coupled to Enzo's
existing operator-split solver framework in the following manner:
\begin{itemize}
\item[(i)] Evolve the radiation implicitly in time [{\tt gFLDSplit}].
\item[(ii)] Evolve the coupled gas energy correction and Hydrogen I
  number density equations implicitly in time with $O(\dt^2)$-accuracy
  [{\tt gFLDSplit}].  
\item[(iii)] Project the dark matter particles onto the finite-volume
  mesh to generate a dark-matter density field $\rho_{dm}$ [Enzo];
\item[(iv)] Solve for the gravitational potential $\phi$ using a
  Poisson equation [Enzo];
\item[(v)] Advect the dark matter particles with the Particle-Mesh
  method [Enzo];
\item[(vi)] Evolve the hydrodynamics equations using an up to
  second-order explicit method, and have the velocity $\vb$ advect
  both the Hydrogen I number density $\mn_{HI}$ and the grey radiation
  field $E$ [Enzo]; 
\item[(vii)] Evolve the gas energy correction and chemical ionization
  states in time using split, linearly-implicit solvers with
  $O(\dt^{\alpha})$-accuracy ($0<\alpha<1$) [Enzo].
\end{itemize}

We note that steps (ii) and (vii) are mutually exclusive.  Only one of
the two is used, but we include them both above to explicitly show the
order of operations.

The implicit solution approach for step (i) is similar to the one from 
\cite{ReynoldsHayesPaschosNorman2009}; here we describe only enough
to point out the available user parameters, and more fully describe
some additional options available in the solver.

In solving the steps (i) and (ii), we first discretize the
equations \eqref{eq:radiation_PDE}, \eqref{eq:cons_energy_correction}
and \eqref{eq:hydrogen_ionization} in space and time before we solve
them computationally.  We use a method of lines approach for the
space-time discretization, wherein we first discretize in space, and
then evolve the resulting system of ODEs in time.  As with the rest of
Enzo, we use a finite-volume spatial discretization, placing all of
our unknowns at the center of each finite-volume cell, and performing
all spatial derivatives through a divergence of face-centered fluxes.


\subsection{Radiation subsystem}
\label{sec:rad_solve}

We discretize the radiation equation \eqref{eq:radiation_PDE} using a
standard two-level $\theta$-method,
\begin{align}
  \label{eq:radiation_PDE_theta}
  E^n - E^{n-1} &- \theta\dt\left(\nabla\cdot\(D\,\nabla E^n\) - \frac{\dot{a}}{a} E^n -
    c\kappa^n E^n + \eta^n\right) \\ 
  \notag
  & - (1-\theta)\dt\left(\nabla\cdot\(D\,\nabla E^{n-1}\) - \frac{\dot{a}}{a} E^{n-1} -
    c\kappa^{n-1} E^{n-1} + \eta^{n-1}\right) = 0,
\end{align}
where $0\le\theta\le 1$ defines the time-discretization, and where we
have assumed that the advective portion of \eqref{eq:radiation_PDE}
has already been taken care of through Enzo's hydrodynamics solver.
Recommended values of $\theta$ are 1 (backwards Euler) and $\frac12$
(trapezoidal, a.k.a.~Crank-Nicolson).  

Whichever $\theta$ value we use (as long as it is nonzero), the
equation \eqref{eq:radiation_PDE_theta} is linearly-implicit in the
time-evolved radiation energy density $E^n$.  We write this in
predictor-corrector form (for ease of boundary condition
implementation), which we will write as
\begin{align}
\label{eq:linear_system}
  J s = b, \qquad E^n = E^{n-1} + s.
\end{align}
We approximately solve this linear equation for the update $s$,
to a tolerance $\delta$,
\begin{align}
\label{eq:linear_system_approx}
  \| J s - b \|_2 \le \delta,
\end{align}
using using a multigrid-preconditioned conjugate gradient iteration.




\subsection{Gas and Chemistry subsystem}
\label{sec:analytic_solve}

As with the radiation equation, we also assume that the advective
portion of \eqref{eq:hydrogen_ionization} is taken care of using
Enzo's hydrodynamics solvers.  Therefore, since the remainder of the
gas energy and chemistry equations is spatially local, the solver for
step (ii) is performed separately on a cell-by-cell basis.  To this
end, we employ a new implicit-time version of the 
{\em quasi-steady-state approximation}.  In this approach, instead of
approximating the solution to the exact ODEs, we exactly solve
approximate ODEs.  Here, if we assume in each equation that all but the
time-evolving variable are held constant throughout the time step, we
could consider the ODE system 
\begin{align}
  \label{eq:energy_correction_qss}
  \partial_t e_c &= -\frac{2\dot{a}}{a}e_c + G(\overline{E},\overline{\mn_{HI}}) - \Lambda(\overline{E},e,\overline{\mn_{HI}}), \\
  \label{eq:hydrogen_qss}
  \partial_t \mn_{HI} &= \alpha^{rec}(\overline{e}) \mn_e \mn_{HII} - \mn_{HI} \Gamma_{HI}^{ph}(\overline{E}),
\end{align}
where $\overline{u}$ denotes a field $u$ that is assumed fixed
throughout a time step.  With this approximation,
\eqref{eq:energy_correction_qss}-\eqref{eq:hydrogen_qss} may be
written as
\begin{align}
  \label{eq:energy_correction_qss2}
  \partial_t e_c &= Q - Pe_c, \\
  \label{eq:hydrogen_qss2}
  \partial_t \mn_{HI} &= a \mn_{HI}^2 + b\mn_{HI} + c,
\end{align}
where $P$, $Q$, $a$, $b$ and $c$ are all constant throughout the time
step.  These may be solved analytically to full accuracy for any time step
size $\dt$.  We write these analytical solvers as
\begin{align}
  \label{eq:energy_correction_qss3}
  e_c(t) &= \text{sol}_e\left(\overline{E},\overline{\mn_{HI}},e_c^{n-1},t\right), \\
  \label{eq:hydrogen_qss3}
  \mn_{HI}(t) &= \text{sol}_{HI}\left(\overline{E},\overline{e},\mn_{HI}^{n-1},t\right),
\end{align}
and couple these together implicitly through defining the nonlinear
equations 
\begin{align}
  \label{eq:energy_correction_iqss}
  f_e(e_c,\mn_{HI}) = e_c^n &- \text{sol}_e\left(\frac{E^{n-1}+E^n}{2},\frac{\mn_{HI}^{n-1}+\mn_{HI}^n}{2},e_c^{n-1},t\right) = 0, \\
  \label{eq:hydrogen_iqss}
  f_{HI}(e_c,\mn_{HI}) = \mn_{HI}^n &- \text{sol}_{HI}\left(\frac{E^{n-1}+E^n}{2},\frac{e^{n-1}+e^n}{2},\mn_{HI}^{n-1},t\right) = 0.
\end{align}

As solutions to these equations can change rather dramatically
(e.g.~ionization states may change by orders of magnitude in a cell in
a single time step), we use a highly robust damped fixed-point
iteration to solve the equations
\eqref{eq:energy_correction_iqss}-\eqref{eq:hydrogen_iqss},
\begin{equation}
  \label{eq:nonlinear_system}
  \begin{split}
  |f_e(e_c,\mn_{HI})| &< 10^{-8}, \\
  |f_{HI}(e_c,\mn_{HI})| &< 10^{-8},
  \end{split}
\end{equation}
at each time step to obtain the updated solution variables $e_c^n$ and
$\mn_{HI}^n$.

We note that if the local thermodynamic equilibrium approximation is
used, then this subsystem only solves for the time-evolved gas energy
correction $e_c^n$.




\subsection{Time-step selection}
\label{sec:dt_selection}

Time steps are chosen adaptively in an attempt to control error in the
calculated solution.  To this end, we first define an heuristic
measure of the time accuracy error in a specific variable $u$ as
\begin{align}
\label{eq:time_error}
  err = \left(\frac1N \sum_{i=1}^N
    \left(\frac{u_i^{n}-u_i^{n-1}}{\omega_i}\right)^p\right)^{1/p}, 
\end{align}
where the weighting vector $\omega$ is given by
\begin{align}
\label{eq:time_weighting}
  \omega_i &= \sqrt{u_i^n u_i^{n-1}} + 10^{-3}, \quad i=1,\ldots,N, \\
  \omega_i &= |e_{c,i} + e_{h,i}| + 10^{-3}, \quad i=1,\ldots,N,
\end{align}
i.e.~we scale the radiation and chemistry change by the geometric mean
of the old and new states, and scale the gas energy change by the new
total gas energy, adding on a floor value of $10^{-3}$ in case any
of the states are too close to zero.  This approach works well when
the internal solution variables are unit-normalized, or at least close
to unit-normalized, since the difference between the old and new
solutions, divided by this weighting factor $\omega$, should give a
reasonable estimate of the number of significant digits that are
correct in the solution. 

With these error estimates \eqref{eq:time_error} for each variable, we
set the new time step size for each subsystem based on the previous
time step size and a user-input tolerance $\tau_{\text{tol}}$ as
\begin{align}
\label{eq:time_estimate}
  \dt^{n} = \frac{\tau_{\text{tol}} \dt^{n-1}}{err}.
\end{align}
Since $E$ and $\{e_c,\mn_{HI}\}$ are evolved separately,
we allow the $\{e_c,\mn_{HI}\}$ solver to subcycle at a faster rate
than the $E$.  We therefore have two time step sizes that we use in
the module,
\begin{align}
\label{eq:FLD_time_estimate}
  \dt_{E}^{n} &= \min\{\dt_{E}^{n},\dt_{CFL}^{n}\}. \\
  \dt_{e,HI}^{n} &= \min\{\dt_{e}^{n},\dt_{HI}^{n},\dt_{E}^{n}\},
\end{align}
where $\dt_{\text{CFL}}$ is the time step size that Enzo's other
routines (e.g.~hydrodynamics) would normally take.  A user may
override these adaptive time step controls with the input parameters 
$\dt_{\text{max}}$ and $\dt_{\text{min}}$. 

We further note that when run in combination with Enzo's hydrodynamics
routines, both modules will limit their maximum time step sizes to the
minimum of $\dt_{\text{E}}$ and $\dt_{\text{CFL}}$.  As a result, in
some physical regimes, the global time step size will be limited based
on the radiation time scale, and in other regimes it will be limited
by the hydrodynamic time scale. 




\subsection{Variable rescaling}
\label{sec:variable_rescaling}

In case Enzo's standard unit non-dimensionalization using 
{\tt DensityUnits}, {\tt LengthUnits} and {\tt TimeUnits} is
insufficient to render the resulting solver values $E$, $e_c$ and
$n_{HI}$ to have nearly unit magnitude, the user may input additional
variable scaling factors to be used inside the {\tt gFLDSplit}
module.  Denoting these user-input values as $s_E$, $s_e$ and
$s_{\mn}$, then we may define the rescaled variables
\begin{align}
\label{eq:variable_rescaling}
  \tilde{E} = E / s_E, \qquad \tilde{e}_c = e_c / s_e, \qquad 
  \tilde{\mn}_{HI} = \mn_{HI} / s_{\mn},
\end{align}
and the {\tt gFLDSplit} module will use $\tilde{E}$, $\tilde{e}_c$ and
$\tilde{\mn}_{HI}$ in its internal routines instead of Enzo's internal
variables $E$, $e_c$ and $\mn_{HI}$.  If the user does not know
appropriate values for these scaling factors {\em a-priori}, a
generally-applicable rule of thumb is to first run their simulation
for a small number of time steps and investigate Enzo's HDF5 output
files to see the magnitude of the values stored internally by Enzo; if
these are far from unit-magnitude, these scaling factors should be
used. 



\subsection{Boundary conditions}
\label{sec:boundary_conditions}

As the radiation equation \eqref{eq:radiation_PDE} is parabolic,
boundary conditions must be supplied on the radiation field $E$.  The
{\tt gFLDSplit} module allows three types of boundary conditions to
be placed on the radiation field:
\begin{itemize}
\item[0.] Periodic,
\item[1.] Dirichlet, i.e.~$E(x,t) = g(x), \; x\in\partial\Omega$, and
\item[2.] Neumann, i.e.~$\nabla E(x,t)\cdot n = g(x), \; x\in\partial\Omega$.
\end{itemize}
In most cases, the boundary condition types (and values of $g$) are
problem-dependent.  When adding new problem types, these conditions
should be set near the bottom of the file {\tt gFLDSplit\_Initialize.C}, 
otherwise these will default to either (a) periodic, or (b) will use
$g=0$, depending on the user input boundary condition type.



\section{Concluding remarks}
\label{sec:conclusions}

We wish to remark that the module is not large (one header
file, 15 C++ files, 6 F90 files), and all files begin with the 
{\tt gFLDSplit} prefix.  While we have strived to ensure that the
module is bug-free, there is still work to be done in enabling
additional physics, including fully implicit Helium/molecular
chemistry and more advanced time-stepping interactions with the rest
of Enzo (especially when ionization sources ``turn on'' abruptly).  

Feedback/suggestions to are welcome.


\bibliography{sources}
\bibliographystyle{siam}
\end{document}
