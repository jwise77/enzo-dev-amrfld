\documentclass[letterpaper,10pt]{article}
\usepackage{geometry}   % See geometry.pdf to learn the layout
                        % options.  There are lots.
\usepackage[latin1]{inputenc}
\usepackage{graphicx}
\usepackage{epstopdf}
\usepackage{amsmath,amsfonts,amssymb}

\DeclareGraphicsRule{.tif}{png}{.png}{`convert #1 `dirname #1`/`basename #1 .tif`.png}

\author{Daniel R. Reynolds}
\title{New time-stepping approach for {\tt gFLDSplit}}

\renewcommand{\(}{\left(}
\renewcommand{\)}{\right)}
\newcommand{\vb}{{\bf v}_b}
\newcommand{\xvec}{{\bf x}}
\newcommand{\Omegabar}{\bar{\Omega}}
\newcommand{\rhob}{\rho_b}
\newcommand{\dt}{\Delta t}
\newcommand{\Eot}{E^{OT}}
\newcommand{\Ef}{E_f}
\newcommand{\sighat}{\hat{\sigma}}
\newcommand{\Fnu}{{\bf F}_{\nu}}
\newcommand{\Pnu}{\overline{\bf P}_{\nu}}
\newcommand{\R}{I\!\!R}
\newcommand{\Rthree}{\R^3}
\newcommand{\eh}{e_h}
\newcommand{\ec}{e_c}
\newcommand{\Edd}{\mathcal F}
\newcommand{\Eddnu}{\Edd_{\nu}}
\newcommand{\mn}{{\tt n}}
\newcommand{\mB}{\mathcal B}
\newcommand{\mC}{{\mathcal C}}
\newcommand{\mL}{{\mathcal L}}
\newcommand{\mD}{{\mathcal D}}
\newcommand{\mDnu}{\mD_{\nu}}
\newcommand{\mCnu}{\mC_{\nu}}
\newcommand{\mLnu}{{\mathcal L}_{\nu}}
\newcommand{\mCe}{\mC_e}
\newcommand{\mLe}{\mL_e}
\newcommand{\mCn}{\mC_{\mn}}
\newcommand{\mLn}{\mL_{\mn}}


\textheight 9truein
\textwidth 6.5truein
\addtolength{\oddsidemargin}{-0.25in}
\addtolength{\evensidemargin}{-0.25in}
\addtolength{\topmargin}{-0.5in}
\setlength{\parindent}{0em}
\setlength{\parskip}{2ex}


\begin{document}
\maketitle


\section{Flux-limited diffusion radiation model}
\label{sec:rad_model}

We begin with the equation for flux-limited diffusion radiative
transfer in a cosmological medium \cite{ReynoldsHayesPaschosNorman2009},
\begin{equation}
\label{eq:radiation_PDE}
  \partial_{t} E + \frac1a \nabla\cdot\(E\vb\) =
    \nabla\cdot\(D\,\nabla E\) - \frac{\dot{a}}{a} E - c\kappa E + 4\pi\eta,
\end{equation}
where here the comoving radiation energy density $E$, emissivity
$\eta$ and opacity $\kappa$ are functions of space and time.  In this
equation, the frequency-dependence of the radiation energy has been
integrated away, under the premise of an assumed radiation energy
spectrum, 
\begin{align}
  \notag
  & E_{\nu}(\nu,\xvec,t) = \tilde{E}(\xvec,t) \chi(\nu), \\
  \notag
  \Rightarrow & \\
  \label{eq:spectrum}
  & E(\xvec,t) = \int_{\nu_{HI}}^{\infty} E_{\nu}(\nu,\xvec,t)\,\mathrm{d}\nu 
    = \tilde{E}(\xvec,t) \int_{\nu_{HI}}^{\infty} \chi(\nu)\,\mathrm{d}\nu,
\end{align}
where $\tilde{E}$ is an intermediate quantity (for analysis) that is
never computed.  We note that if the assumed spectrum is the Dirac
delta function, $\chi(\nu) = \delta_{\nu_{HI}}(\nu)$, $E$ is a
monochromatic radiation energy density at the ionization threshold of
HI, and the $-\frac{\dot{a}}{a}E$ term (obtained through integration
by parts of the redshift term
$\frac{\dot{a}}{a}\partial_{\nu}E_{\nu}$) is omitted from
\eqref{eq:radiation_PDE}. Similarly, the emissivity function
$\eta(\xvec,t)$ relates to the true emissivity
$\eta_{\nu}(\nu,\xvec,t)$ by the formula
\begin{equation}
\label{eq:emissivity}
  \eta(\xvec,t) = \int_{\nu_{HI}}^{\infty}\eta_{\nu}(\nu,\xvec,t)\,\mathrm{d}\nu.
\end{equation}

The function $D$ in the above equation \eqref{eq:radiation_PDE} is
the {\em flux-limiter} that depends on $E$, $\nabla E$ and the 
opacity $\kappa$ (see
\cite{HayesNorman2003,ReynoldsHayesPaschosNorman2009}),  
\[
   D(E) = \text{diag}\( D_1(E),\, D_2(E),\, D_3(E) \),
\]
where the directional limiters $D_i(E)$ are chosen to be one of 
\begin{align}
  \label{eq:ratLP_limiter}
   D_i(E) &= \frac{c(2\kappa+R_i)}{\omega(6\kappa^2+3\kappa R_i + R_i^2)}, \\
  \label{eq:Larsen_n2_limiter}
   D_i(E) &= \frac{c}{\left((3\kappa)^2 + R_i^2\right)^{1/2}},  \\
  \label{eq:no_limiter}
   D_i(E) &= \frac{c}{3\kappa}, \\
  \label{eq:Zeus_limiter}
   D_i(E) &= \frac{c(2\kappa+R_i)}{6\kappa^2+3\kappa R_i+R_i^2}, \\
  \label{eq:LP_limiter}
   D_i(E) &= \frac{c \tanh\left(\frac{R_i}{\kappa}\right)-c \frac{\kappa}{R_i}}{\omega R_i}.
\end{align}
In all of the above, the ``effective albedo'' is given by
\begin{equation}
\label{eq:albedo}
  \omega = \frac{(4 \sigma_{SB} T^4)}{c E},
\end{equation}
and the limiter factor $R_i$ is given by either
\begin{align}
  \label{eq:R_zeus}
   R_i &= \max\left\{\frac{|\partial_i E|}{\omega E},
     10^{-20}\right\}, \qquad\text{[Levermore-Pomraning limiters,
     \eqref{eq:ratLP_limiter} \& \eqref{eq:LP_limiter}]}, \\
   R_i &= \max\left\{\frac{|\partial_i E|}{E}, 10^{-20}\right\},
     \qquad\text{[all others]}.
\end{align}



\section{LTE couplings}
\label{sec:lte_model}

For problems in which chemical ionization is unimportant, we may
assume that the gas is in local thermodynamic equilibrium.  In this
case we do not need to couple the radiation energy to a model for
chemical ionization, we therefore couple the radiation to the specific
gas energy equation, 
\begin{align}
  \label{eq:cons_energy}
  \partial_t e + \frac1a\vb\cdot\nabla e &=
    - \frac{2\dot{a}}{a}e
    - \frac{1}{a\rhob}\nabla\cdot\left(p\vb\right) 
    - \frac1a\vb\cdot\nabla\phi + G - \Lambda.
\end{align}
All but the final two terms in \eqref{eq:cons_energy} are already
handled by Enzo's existing hydrodynamics solver infrastructure.  In
this module, we therefore consider a specific energy correction equation 
\begin{align}
  \label{eq:cons_energy_correction}
  \partial_t e_c &= -\frac{2\dot{a}}{a}e_c + G - \Lambda,
\end{align}
that we use to correct Enzo's original gas energy to include radiation
couplings.  Here $G$ is the local heating rate,
\begin{align}
\label{eq:G_LTE}
  G &= \frac{c \kappa}{\rhob} E,
\end{align}
and $\Lambda$ corresponds to the local cooling rate,
\begin{align}
\label{eq:Lambda_LTE}
  \Lambda = \frac{4\pi}{\rhob} \eta.
\end{align}
The user-defined energy mean opacity $\kappa$ is
spatially-homogeneous, and is given by the formula
\begin{align}
\label{eq:opacityE}
  \kappa = C_0 \left(\frac{\rhob}{C_1}\right)^{C_2}
    \left(\frac{T}{C_3}\right)^{C_4} 
\end{align}
(the constants $C_0\to C_4$ are input by the user), and $\eta$ is
a black-body emissivity given by 
\begin{align}
\label{eq:etaBB}
  \eta = \frac{\kappa_P\,\sigma_{SB}\,T^4}{\pi},
\end{align}
where the user-defined Planck-mean opacity $\kappa_P$ is defined
similarly to the energy-mean opacity \eqref{eq:opacityE},
$\sigma_{SB}$ is the Stefan-Boltzmann constant [$5.6704\times 10^{-5}$
erg s$^{-1}$ cm$^{-2}$ K$^{-4}$], and $T$ is the gas temperature [K].



\section{Chemistry-dependent couplings}
\label{sec:chem_model}

In general, radiation calculations in Enzo are used in simulations
where chemical ionization states are important.  For these situations, 
we couple the radiation equation \eqref{eq:radiation_PDE} with
equations for both the specific gas energy correction and the
ionization dynamics of Hydrogen,
\begin{align}
  \notag
  \partial_t e_c &= -\frac{2\dot{a}}{a}e_c + G - \Lambda, \\
  \label{eq:hydrogen_ionization}
  \partial_t \mn_{HI} + \frac{1}{a}\nabla\cdot\(\mn_{HI}\vb\) &=
    \alpha^{rec} \mn_e \mn_{HII} - \mn_{HI} \Gamma_{HI}^{ph}. 
\end{align}
Here, $\mn_{HI}$ is the comoving Hydrogen I number density.  The
recombination rate $\alpha^{rec}$ may be chosen as either the case-B
recombination rate, 
\begin{equation}
\label{eq:alphaB}
\alpha^{rec} = 2.753\times 10^{-14} \left(\frac{3.15614\times 10^5}{T}\right)^{3/2} 
                   \left(1+\left(\frac{3.15614\times 10^5}{2.74\, T}\right)^{0.407}\right)^{-2.242} 
\end{equation}
or the case-A rate found in Enzo's chemistry rate lookup tables.

In this model, the gas heating and cooling rates are
chemistry-dependent, 
\begin{align}
  \label{eq:G_nLTE}
  G &= \frac{c\,E\,\mn_{HI}}{\rhob} 
    \left[\int_{\nu_{HI}}^{\infty} \sigma_{HI}\, \chi_E
    \left(1-\frac{\nu_{HI}}{\nu}\right)\, d\nu\right] \bigg/
    \left[\int_{\nu_{HI}}^{\infty} \chi_E d\nu\right], \\
\label{eq:Lambda_nLTE}
  \Lambda &= \frac{\mn_e}{\rhob}\bigg[\text{ce}_{HI}\, \mn_{HI} 
  + \text{ci}_{HI}\, \mn_{HI} + \text{re}_{HII}\, \mn_{HII} + \text{brem}\,
  \mn_{HII} \\
  \notag &\qquad+ \frac{m_h}{\rho_{units}\, a^3} \left(\text{comp}_1\, (T-\text{comp}_2) 
    + \text{comp}_{X}\, (T-\text{comp}_{T})\right) \bigg].
\end{align}
The temperature-dependent cooling rates
$\text{ce}_{HI}$, $\text{ci}_{HI}$, $\text{re}_{HII}$, $\text{brem}$,
$\text{comp}_1$, $\text{comp}_2$, $\text{comp}_{X}$ and
$\text{comp}_{T}$ are all taken from Enzo's built-in rate tables.

Moreover, the frequency-integrated opacity is now chemistry-dependent,
\begin{equation}
\label{eq:opacityHI}
  \kappa \ = \ 
  \left[\int_{\nu_{HI}}^{\infty} \kappa_{\nu}\,E_{\nu}\,d\nu\right] \bigg/
  \left[\int_{\nu_{HI}}^{\infty} E_{\nu}\,d\nu\right] \ = \ 
  \left[\mn_{HI} \int_{\nu_{HI}}^{\infty}
    \chi_E\,\sigma_{HI}\,d\nu\right] \bigg/
  \left[\int_{\nu_{HI}}^{\infty} \chi_E\,d\nu\right],
\end{equation}
where these integrals with the assumed radiation spectrum $\chi(\nu)$
handle the change from the original frequency-dependent radiation
equation to the integrated grey radiation equation. 




\section{Numerical solution approach}
\label{sec:solution_approach}

We solve these models in an implicit-time fashion, coupling 
\eqref{eq:radiation_PDE}, \eqref{eq:cons_energy_correction} and
possibly \eqref{eq:hydrogen_ionization} together to form a large
system of nonlinear PDEs that must be solved in each time step to find
the updated solution.  This solve is coupled to Enzo's existing
operator-split solver framework in the following manner:
\begin{itemize}
\item[(i)] Evolve the radiation, gas energy correction and Hydrogen I
  number density implicitly in time [{\tt gFLDProblem}].
\item[(ii)] Project the dark matter particles onto the finite-volume
  mesh to generate a dark-matter density field $\rho_{dm}$ [Enzo];
\item[(iii)] Solve for the gravitational potential $\phi$ using a
  Poisson equation [Enzo];
\item[(iv)] Advect the dark matter particles with the Particle-Mesh
  method [Enzo];
\item[(v)] Evolve the hydrodynamics equations using an up to
  second-order explicit method, and have the velocity $\vb$ advect
  both the Hydrogen I number density $\mn_{HI}$ and the grey radiation
  field $E$ [Enzo]; 
\end{itemize}

The implicit solution approach for step (i) is described in detail in 
\cite{ReynoldsHayesPaschosNorman2009}; here we describe only enough
to point out the available user parameters, and more fully describe
some additional options available in the solver.

\subsection{Time discretizations}
\label{sec:iqss}

We must first discretize the equations \eqref{eq:radiation_PDE},
\eqref{eq:cons_energy_correction} and \eqref{eq:hydrogen_ionization}
in space and time before we solve them computationally.  We use a
method of lines approach for the space-time discretization, wherein we
first discretize in space, and then evolve the resulting system of
ODEs in time.  As with the rest of Enzo, we use a finite-volume
spatial discretization, placing all of our unknowns at the center of
each finite-volume cell, and performing all spatial derivatives
through a divergence of face-centered fluxes.  

However, this module allows two different time discretizations of our
ODE system \eqref{eq:radiation_PDE}, \eqref{eq:cons_energy_correction}
and \eqref{eq:hydrogen_ionization}.  The first, described in
\cite{ReynoldsHayesPaschosNorman2009}, employs a two-level
$\theta$-method for the time discretization, and results in the
equations to compute the time-evolved solution $(E^n,e_c^n,\mn_{HI}^n)$,
\begin{align}
  \label{eq:radiation_PDE_theta}
  E^n - E^{n-1} &- \theta\dt\left(\nabla\cdot\(D\,\nabla E^n\) - \frac{\dot{a}}{a} E^n -
    c\kappa^n E^n + 4\pi\eta^n\right) \\ 
  \notag
  & - (1-\theta)\dt\left(\nabla\cdot\(D\,\nabla E^{n-1}\) - \frac{\dot{a}}{a} E^{n-1} -
    c\kappa^{n-1} E^{n-1} + 4\pi\eta^{n-1}\right) = 0, \\ 
  \label{eq:energy_correction_theta}
  e_c^n - e_c^{n-1} &- \theta\dt\left(-\frac{2\dot{a}}{a}e_c^{n} + G^{n} -
    \Lambda^{n}\right) - (1-\theta)\dt\left(-\frac{2\dot{a}}{a}e_c^{n-1} + G^{n-1} -
    \Lambda^{n-1}\right) = 0, \\
  \label{eq:hydrogen_theta}
  \mn_{HI}^n - \mn_{HI}^{n-1} &-
    \theta\dt\left(\alpha^{rec,n} \mn_e^{n} \mn_{HII}^{n} -
      \mn_{HI}^{n} \Gamma_{HI}^{ph,n}\right) -
    (1-\theta)\dt\left(\alpha^{rec,n-1} \mn_e^{n-1} \mn_{HII}^{n-1} -
      \mn_{HI}^{n-1} \Gamma_{HI}^{ph,n-1}\right) = 0,
\end{align}
where $0\le\theta\le 1$ defines the time-discretization, and where we
have assumed that the advective portions of \eqref{eq:radiation_PDE}
and \eqref{eq:hydrogen_ionization} have already been taken care of
through Enzo's hydrodynamics solver.  Recommended values of $\theta$
are 1 (backwards Euler) and $\frac12$ (trapezoidal,
a.k.a.~Crank-Nicolson).  Benefits of this approach include:
\begin{itemize}
\item Standard, easily understandable approach,
\item Unified treatment of all variables in implicit system,
\item Ease of analytical Jacobians for efficient solution of the
  resulting implicit systems.
\item Theoretical asymptotic accuracy of $O(\dt)$ for $\theta=1$ and
  $O(\dt^2)$ for $\theta=\frac12$. 
\end{itemize}
However, a key drawback of this approach is that it knows nothing
about the inherent constraints on the solution variables, i.e.~$E\ge
0$, $e>0$ and $\mn_{HI}\ge 0$.  As a result, in problems with
rapidly-varying chemical states in which $\mn_{HI}$ in a
newly-irradiated finite volume cell should change by multiple orders 
of magnitude in a single time step, these simple linear or quadratic
interpolations in time can result in negative solution values due to
time discretization error.  Therefore, in using the above system 
\eqref{eq:radiation_PDE_theta}-\eqref{eq:hydrogen_theta}, it is
imperative that the user allow {\em very conservative} time step sizes
to be used (see section \ref{sec:dt_selection}).

As a result, we have also implemented a second approach to time
discretization that should be much more robust, while attempting to
remain as accurate as possible.  This approach is based on a new
implicit-time version of the {\em quasi-steady-state approximation}.
In this approach, instead of approximating the solution to the exact
ODEs, we exactly solve approximate ODEs.  Here, if we assume in each
equation that all but the time-evolving variable are held constant
throughout the time step, we could consider the ODE system
\begin{align}
  \label{eq:energy_correction_qss}
  \partial_t e_c &= -\frac{2\dot{a}}{a}e_c + G(\overline{E},\overline{\mn_{HI}}) - \Lambda(\overline{E},e,\overline{\mn_{HI}}), \\
  \label{eq:hydrogen_qss}
  \partial_t \mn_{HI} &= \alpha^{rec}(\overline{e}) \mn_e \mn_{HII} - \mn_{HI} \Gamma_{HI}^{ph}(\overline{E}),
\end{align}
where $\overline{u}$ denotes a field $u$ that is assumed fixed
throughout a time step.  With this approximation,
\eqref{eq:energy_correction_qss}-\eqref{eq:hydrogen_qss} may be
written as
\begin{align}
  \label{eq:energy_correction_qss2}
  \partial_t e_c &= Q - Pe_c, \\
  \label{eq:hydrogen_qss2}
  \partial_t \mn_{HI} &= a \mn_{HI}^2 + b\mn_{HI} + c,
\end{align}
where $P$, $Q$, $a$, $b$ and $c$ are constant throughout the time
step.  These may be solved analytically to full accuracy for any time
step size $\dt$.  We write these analytical solvers as
\begin{align}
  \label{eq:energy_correction_qss3}
  e_c(t) &= \text{sol}_e\left(\overline{E},\overline{\mn_{HI}},e_c^{n-1},t\right), \\
  \label{eq:hydrogen_qss3}
  \mn_{HI}(t) &= \text{sol}_{HI}\left(\overline{E},\overline{e},\mn_{HI}^{n-1},t\right).
\end{align}
Since such an approach does not work well for PDEs (e.g.~equation
\eqref{eq:radiation_PDE}), and since the radiation equation constraint
is typically less problematic than the gas energy and chemistry, we
still use the $\theta$-method to discretize $E$.  Putting these
together, our second approach to time discretization uses the
equations 
\begin{align}
  \label{eq:radiation_PDE_iqss}
  E^n - E^{n-1} &- \theta\dt\left(\nabla\cdot\(D\,\nabla E^n\) - \frac{\dot{a}}{a} E^n -
    c\kappa^n E^n + 4\pi\eta^n\right) \\ 
  \notag
  & - (1-\theta)\dt\left(\nabla\cdot\(D\,\nabla E^{n-1}\) - \frac{\dot{a}}{a} E^{n-1} -
    c\kappa^{n-1} E^{n-1} + 4\pi\eta^{n-1}\right) = 0, \\ 
  \label{eq:energy_correction_iqss}
  e_c^n &- \text{sol}_e\left(\frac{E^{n-1}+E^n}{2},\frac{\mn_{HI}^{n-1}+\mn_{HI}^n}{2},e_c^{n-1},t\right) = 0, \\
  \label{eq:hydrogen_iqss}
  \mn_{HI}^n &- \text{sol}_{HI}\left(\frac{E^{n-1}+E^n}{2},\frac{e^{n-1}+e^n}{2},\mn_{HI}^{n-1},t\right) = 0.
\end{align}
Benefits of this approach (as opposed to the $\theta$ method) include:
\begin{itemize}
\item Robust solvers for gas energy and chemistry that can {\em never}
  result in negative solution values,
\item Fully implicit formulation in which all updated solution values
  $E^n$, $e_c^n$ and $\mn_{HI}^n$ depend on one another,
\item Asymptotic time accuracy of $O(\dt^2)$ at $\theta=\frac12$,
  and remarkably good accuracy at even large $\dt$.
\end{itemize}
Unfortunately, however, the use of these solvers makes analytical
derivation of Jacobians very difficult, so they must be approximated,
requiring additional computation per time step.


\subsection{Solver initial guess}
\label{sec:initial_guess}

Performance of the Newton iteration is highly influenced by a good
initial guess: a good guess results in a robust and efficient method,
while a poor guess may take much longer to converge (if at all).  We
therefore have a number of options that may be used to determine the
initial guess $U_0$:
\begin{itemize}
\item[0.] Use the previous solution 
  $U_0 = U^{n-1}$.
\item[1.] Use an explicit predictor: $U_0 = U^{n-1} + \dt
  \mathcal L(U^{n-1})$, where $\mathcal L$ comprises all of the
  spatially-local physics (i.e.~no diffusion).
\item[2.] Use an explicit predictor: $U_0 = U^{n-1} + \dt
  \mathcal R(U^{n-1})$, where $\mathcal R$ comprises all of the
  physics.
\item[3.] Use a partial explicit predictor: $U_0 = U^{n-1} + \frac{\dt}{10}
  \mathcal L(U^{n-1})$.
\item[4.] Use a partial explicit predictor: $U_0 = U^{n-1} + \frac{\dt}{10}
  \mathcal R(U^{n-1})$.
\item[5.] Use an analytic predictor of spatially-local physics: $U_0
  = (E^{n-1},\text{sol}_e,\text{sol}_{HI})$. 
\end{itemize}

\subsection{Time-step selection}
\label{sec:dt_selection}

Consider the equations ()-() as a system of ODEs,
\begin{align*}
  \partial_t E &= r_E(E,e_c,\mn_{HI}), \\
  \partial_t e_c &= r_e(E,e_c,\mn_{HI}), \\
  \partial_t \mn_{HI} &= r_{HI}(E,e_c,\mn_{HI}),
\end{align*}
Then one may measure the relative rate of change in each variable as
\begin{align*}
  r_E &= \left\|\frac{r_E}{E}\right\|_p \approx \left\|\frac{r_E}{E + atol}\right\|_p, \\
  r_e &= \left\|\frac{r_e}{e}\right\|_p \approx \left\|\frac{r_e}{e + atol}\right\|_p, \\
  r_{HI} &= \left\|\frac{r_{HI}}{\mn_{HI}}\right\|_p \approx \left\|\frac{r_{HI}}{\mn_{HI} + atol}\right\|_p,
\end{align*}
i.e.~we scale the time rate of change in each variable by the solution
value, with an additional scaling floor ($atol$) included to account
for situations in which the current solution is well below the
dynamical scale.  We choose $atol=10^{-3}$, which assumes that the
solution components are at least close to unit-normalized.
We note the use of a $p$-norm,
\[
   \|u\|_p = \begin{cases}
     \left(\frac1N \sum u_i^p\right)^{1/p}, & \text{if}\;\; p > 0, \\
     \max |u_i|, & \text{if}\;\; p = 0, \\
     \text{undefined}, & \text{if}\;\; p < 0,
   \end{cases}
\]
where the vector $u$ has $N$ components.

Alternatively, if we are using one of the quasi-steady-state solvers,
we consider the ODEs
\begin{align*}
  e_c(t) = &= \text{sol}_e(\overline{E},e_c^n,\overline{\mn}_{HI},t-t^n), \\
  \mn_{HI}(t) = &= \text{sol}_{HI}(\overline{E},\overline{e},\mn_{HI}^n,t-t^n),
\end{align*}
in which case the relative rate of change in these variables may be
estimated using the previous time step size $\dt^n$ as
\begin{align*}
  r_e &\approx
  \left\|\frac{\text{sol}_e(\overline{E},e_c^n,\overline{\mn}_{HI},\dt^n)}{\dt^n(e^n + atol)}\right\|_p, \\
  r_{HI} &\approx
  \left\|\frac{\text{sol}_{HI}(\overline{E},\overline{e},\mn_{HI}^n,\dt^n) - \mn_{HI}^n}{\dt^n(\mn_{HI}^n + atol)}\right\|_p,
\end{align*}
where we do not subtract the analytical solution for $e_c^n$ since
this is only a correction, and hence $e_c^n=0$ in this time step.

Once we have the estimated rates of change, we then set the upcoming
time step size based on the desired change in a variable per step,
$\tau_i$, as 
\begin{align*}
  \dt_E &= \frac{r_E}{\tau_E}, \\
  \dt_e &= \frac{r_e}{\tau_e}, \\
  \dt_{HI} &= \frac{r_{HI}}{\tau_{HI}}.
\end{align*}
The final estimated time step size at the step $n$ is given as the
minimum of the three,
\begin{align}
\label{eq:FLD_time_estimate}
  \dt_{\text{FLD}}^n = \min\{\dt_{E},\dt_e,\dt_{HI}, (1.1)\dt_{\text{FLD}}^{n-1}\}. 
\end{align}
The user may override this adaptive time step with the inputs
$\dt_{\text{max}}$ and $\dt_{\text{min}}$. 

We further note that when run in combination with Enzo's hydrodynamics
routines, both modules will limit their maximum time step sizes to the
minimum of $\dt_{\text{FLD}}$ and $\dt_{\text{CFL}}$, where
$\dt_{\text{CFL}}$ is the time step size that Enzo's other routines
would normally take.  As a result, in some physical regimes, the
global time step size will be limited based on the
radiation/ionization/heating time scale, and in other regimes it will
be limited by the hydrodynamic time scale.




\subsection{Variable rescaling}
\label{sec:variable_rescaling}

In case Enzo's standard unit non-dimensionalization with 
{\tt DensityUnits}, {\tt LengthUnits} and {\tt TimeUnits} is
insufficient to render the resulting solver values $E$, $e_c$ and
$n_{HI}$ to have nearly unit magnitude, the user may input additional
variable scaling factors to be used inside the {\tt gFLDProblem}
module.  Denoting the user-input values $s_E$, $s_e$ and $s_{\mn}$,
then all solvers within this module will rescale Enzo's internal
variables to 
\begin{align}
\label{eq:variable_rescaling}
  \tilde{E} = E / s_E, \qquad \tilde{e}_c = e_c / s_e, \qquad \tilde{\mn}_{HI} = \mn_{HI} / s_{\mn},
\end{align}
and use these rescaled values $\tilde{E}$, $\tilde{e}_c$ and
$\tilde{\mn}_{HI}$ instead of Enzo's values $E$, $e_c$ and
$\mn_{HI}$.  If the user does not know appropriate values for these
scaling factors {\em a-priori}, a generally-applicable rule of thumb
is to first run their simulation for a small number of time steps and
investigate Enzo's HDF5 output files to see the magnitude of values
stored internally by Enzo; if these are far from unit-magnitude, these 
scaling factors should be used.



\bibliography{sources}
\bibliographystyle{siam}
\end{document}

