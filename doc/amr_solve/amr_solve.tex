\documentclass[letterpaper,10pt]{article}
\usepackage{geometry}   % See geometry.pdf to learn the layout
                        % options.  There are lots.
\usepackage[latin1]{inputenc}
\usepackage{graphicx}
\usepackage{epstopdf}
\usepackage{amsmath,amsfonts,amssymb}

\usepackage{thumbpdf}   
\usepackage[usenames,pdftex]{color} % black,white,red,green,blue,cyan,magenta,yellow
\pdfcompresslevel=1  
\def\myfig#1#2{\pdfimage height #2 {#1.pdf}}
\def\myfigpdf#1#2{\pdfimage height #2 {#1.pdf}}
\def\myfigpng#1#2{\pdfimage height #2 {#1.png}}

\DeclareGraphicsRule{.tif}{png}{.png}{`convert #1 `dirname #1`/`basename #1 .tif`.png}

\renewcommand{\(}{\left(}
\renewcommand{\)}{\right)}
\newcommand{\vb}{{\bf v}_b}
\newcommand{\xvec}{{\bf x}}
\newcommand{\Omegabar}{\bar{\Omega}}
\newcommand{\rhob}{\rho_b}
\newcommand{\dt}{\Delta t}
\newcommand{\dx}{\Delta x}
\newcommand{\Eot}{E^{OT}}
\newcommand{\Ef}{E_f}
\newcommand{\sighat}{\hat{\sigma}}
\newcommand{\Fnu}{{\bf F}_{\nu}}
\newcommand{\Pnu}{\overline{\bf P}_{\nu}}
\newcommand{\R}{I\!\!R}
\newcommand{\Rthree}{\R^3}
\newcommand{\eh}{e_h}
\newcommand{\ec}{e_c}
\newcommand{\Edd}{\mathcal F}
\newcommand{\Eddnu}{\Edd_{\nu}}
\newcommand{\mn}{{\tt n}}
\newcommand{\mB}{\mathcal B}
\newcommand{\mC}{{\mathcal C}}
\newcommand{\mL}{{\mathcal L}}
\newcommand{\mD}{{\mathcal D}}
\newcommand{\mDnu}{\mD_{\nu}}
\newcommand{\mCnu}{\mC_{\nu}}
\newcommand{\mLnu}{{\mathcal L}_{\nu}}
\newcommand{\mCe}{\mC_e}
\newcommand{\mLe}{\mL_e}
\newcommand{\mCn}{\mC_{\mn}}
\newcommand{\mLn}{\mL_{\mn}}
\newcommand{\amrsolve}{{\tt AMRSolve} }
\newcommand{\enzo}{{\tt Enzo} }
\newcommand{\hypre}{{\tt HYPRE} }
\newcommand{\cpp}{{\tt C++} }


\textheight 9truein
\textwidth 6.5truein
\addtolength{\oddsidemargin}{-0.25in}
\addtolength{\evensidemargin}{-0.25in}
\addtolength{\topmargin}{-0.5in}
\setlength{\parindent}{0em}
\setlength{\parskip}{2ex}


\author{Daniel R. Reynolds and James O. Bordner}
\title{\amrsolve User Guide}


\begin{document}
\maketitle

\section{Introduction}
\label{sec:intro}

The \amrsolve package is an \enzo add-on that strives to achieve two
goals: (a) provide a smoother approach for iteration over grids in an
\enzo AMR hierarchy, both within and between levels, and within and
between MPI tasks, and (b) to provide a straightforward interface for
defining linear systems on block-structured adaptive mesh hierarchies
and solving them using iterative linear solvers provided by the \hypre
solver library.  The first of these alone allows \amrsolve to
significantly benefit \enzo development of new physics modules that
require inter-grid communication, if only due to the ease of
development.  However, the combination of the two goals of \amrsolve
has allowed us to construct two \enzo solvers for implicit systems
defined on AMR grids: self-gravity and flux-limited-diffusion
radiation transport.

In this user guide, we first provide a brief overview of the \amrsolve
package structure, including a description of its suite of \cpp
classes and iterators.  We then focus on the self-gravity solver, 
{\tt AMRGravitySolve}. We will conclude with a discussion of the FLD
radiation transport solver, {\tt AMRFLDSplit}. 


\section{\amrsolve}
\label{sec:amrsolve}

The \amrsolve package contains a relatively large number of classes,
providing a range of utilities from description of the \enzo scalar
type to MPI communicator wrappers, to HYPRE solver interfaces.  All
\amrsolve package routines are held in the \enzo directory 
{\tt src/amr\_solve/}, which are symbolically linked during \enzo
compilation to the {\tt src/enzo/} directory.  We discuss each of
these \cpp classes below, focusing more on those utilities that
interface most closely with \enzo.
\begin{itemize}
\item {\tt AMRsolve\_Domain} -- small class containing the problem
  dimensionality and root-grid extents.
\item {\tt AMRsolve\_Hierarchy} -- this class represents {\tt Enzo}'s
  AMR hierarchy.  However, unlike the \enzo hierarchy structures, this
  class comes with friend classes that provide an invaluable set of
  iterators throughout the hierarchy:
  \begin{itemize}
  \item {\tt ItHierarchyGridsLocal} -- iterator class for visiting all
    task-local grids within an \\
    {\tt AMRsolve\_Hierarchy} object.
  \item {\tt ItHierarchyGridsAll} -- iterator class for visiting all
    grids within an {\tt AMRsolve\_Hierarchy} object (akin to the
    linked-list iteration type through the \enzo hierarchy).
  \item {\tt ItHierarchyLevels} -- iterator class for visiting all
    levels in an an {\tt AMRsolve\_Hierarchy} object, running from
    coarsest to finest.
  \end{itemize}
  An {\tt AMRsolve\_Hierarchy} object is set up within \enzo by
  ``attaching'' it to an \enzo {\tt LevelArray}.  There are currently
  two such routines, one for gravity problems 
  ({\tt enzo\_attach\_grav}) and the other for radiation problems 
  ({\tt enzo\_attach\_fld}).  Upon solution of the necessary
  gravity/radiation problem, the {\tt AMRsolve\_Hierarchy} object may
  be detached from enzo via the {\tt enzo\_detach} routine, allowing
  for adaptivity of the mesh before re-attachment.
\item {\tt AMRsolve\_Level} -- small class for encapsulating a single
  level of an AMR hierarchy.  Again, the true utility in this class is
  it's friend class iterators:
  \begin{itemize}
  \item {\tt ItLevelGridsLocal} -- iterator class for visiting all
    task-local grids in an {\tt AMRsolve\_Level} object.
  \item {\tt ItLevelGridsAll} -- iterator class for visiting all
    grids in an {\tt AMRsolve\_Level} object.
  \end{itemize}
\item {\tt AMRsolve\_Grid} -- class for representing a grid patch in
  an AMR hierarchy.  Each grid is an orthogonal grid of zones in d
  dimensions.  These grids contain a small amount of data for
  interfacing with \hypre solvers, as well as descriptors of the
  corresponding \enzo grid object.  However, these grids do store
  pointers to some of the corresponding \enzo data fields that are
  used in solving up gravity and radiation problems.  The true benefit
  of this class is not to merely reproduce a descriptor of each \enzo
  grid object; instead it comes from the suite of associated
  iterators, that obviate the need to iterate over a full hierarchy to
  determine inter-grid relationships:
  \begin{itemize}
  \item {\tt id\_parent\_} -- this member points to the parent {\tt
    AMRsolve\_Grid}.
  \item {\tt ItGridNeighbors} -- this class iterates over all
    neighboring grids of an {\tt AMRsolve\_Grid}.
  \item {\tt ItGridChildren} -- this class iterates over all child
    grids of an {\tt AMRsolve\_Grid}.
  \end{itemize}
\item {\tt AMRsolve\_Faces} -- each {\tt AMRsolve\_Grid} object has a
  corresponding {\tt AMRsolve\_Faces} object, that identifies what the
  boundaries of the grid are adjacent to (e.g.~coarser cells, finer
  cells, external boundary).
\item {\tt AMRsolve\_Mpi} -- small wrapper class containing the MPI
  communicator, communicator size, and MPI task index.
\item {\tt AMRsolve\_Parameters} -- small class for storing and
  accessing run-time parameters using key-value pairs.
\item {\tt AMRsolve\_Point} -- small class for representing point
  masses (used in some self-gravity tests).
\item {\tt AMRsolve\_Problem} -- small class for setting up
  self-gravity test problems.
\item {\tt AMRsolve\_scalar.h} -- file to define a 'Scalar' type (as
  opposed to re-defining 'float').
\item {\tt AMRsolve\_Hypre\_Grav} -- class for interfacing between the
  \enzo routine {\tt AMRGravitySolve} and \hypre, to perform scalable,
  self-consistent self-gravity solves over a full AMR hierarchy.  This
  will be discussed in further detail in Section
  \ref{sec:AMRGravitySolve}.
\item {\tt AMRsolve\_Hypre\_FLD} -- class for interfacing between the
  \enzo class {\tt AMRFLDSplit} and \hypre, to perform scalable,
  self-consistent flux-limited-diffusion radiation transport solves
  over a full AMR hierarchy.  This will be discussed in further detail
  in Section \ref{sec:AMRFLDSplit}.
\item {\tt AMRsolve\_HG\_Prec} -- class to provide a
  ``hierarchical-grid'' preconditioner for fully implicit linear
  systems posed on block-structured AMR meshes.  This preconditioner
  is designed to be plugged into a \hypre iterative linear solver,
  such as BiCGStab or GMRES, and aids in constructing a scalable
  solver for elliptic (gravity) or parabolic (FLD) linear systems.
\end{itemize}
In order to use the \amrsolve infrastructure (and it's associated
solvers), both of the compile-time options {\tt AMR\_SOLVE} and 
{\tt USE\_HYPRE} must be enabled.


\section{The {\tt AMRGravitySolve} Routine}
\label{sec:AMRGravitySolve}

The {\tt AMRGravitySolve} routine provides an \enzo interface to the
{\tt AMRsolve\_Hypre\_Grav} class within {\tt AMRSolve}.  As with the
existing split gravity solver ({\tt CommunicationSolvePotential})
within \enzo, {\tt AMRGravitySolve} is called within 
{\tt PrepareDensityField}.

When using {\tt AMRGravitySolve}, the self-gravity equation
\begin{equation}
\label{eq:self_gravity}
  \nabla^2 \phi = \frac{4\pi G}{a}(\rho_b + \rho_{dm} + <\!\!\rho_0\!\!>)
\end{equation}
is set up and solved over the AMR hierarchy defined from the root grid
level to the level input to the {\tt PrepareDensityField} function.
Since {\tt PrepareDensityField} is called within {\tt EvolveLevel},
i.e.~on each level in the AMR hierarchy, then {\tt AMRGravitySolve}
will be called on successively deeper or shallower AMR hierarchies as
the W-cycle through the levels is traversed.  The Laplacian is
discretized using a standard second-order accurate finite difference
approximation within each grid patch, with piecewise-constant
interpolation to bridge coarse-fine interface boundaries.

In solving the system \eqref{eq:self_gravity}, {\tt AMRGravitySolve}
will use one of three solver algorithms: GMRES, BiCGStab, or FAC.  The
first two of these are Krylov iterative solver algorithms, while the
third is a multigrid variant called the ``Fast Adaptive Composite''
method \cite{McCormick1989}.  Of these three methods, only BiCGStab
and GMRES are recommended, as the FAC algorithm seems less robust.

When using either of the Krylov linear solvers, the linear system may
be preconditioned using the multigrid-like hierarchical grid
preconditioner defined by the \amrsolve class {\tt AMRsolve\_HG\_Prec}
mentioned above.  

The available input parameters that control the behavior of 
{\tt AMRGravitySolve}, with available options specified in \{\}, and
the default value specified in [], include:
\begin{itemize}
\item {\tt SelfGravityConsistent} \{0, 1\; [0]\} -- choice between
  {\tt CommunicationSolvePotential} (0) and \\
  {\tt AMRGravitySolve} (1) within {\tt PrepareDensityField.C};
\item {\tt AMRGravitySolve\_solver} \{0, 1, 2\; [0]\} -- choice of
  linear solver algorithm, where 1 is GMRES, 2 is FAC, 0 is BiCGStab;
\item {\tt AMRGravitySolve\_maxit} \{$>0$\; [200]\} -- maximum number
  of iterations allowed by linaer solver;
\item {\tt AMRGravitySolve\_useprec} \{0, 1\; [1]\} -- flag to enable
  the {\tt AMRsolve\_HG\_Prec} preconditioner;
\item {\tt AMRGravitySolve\_precmaxit} \{$>0$\; [1]\} -- maximum
  number of preconditioning iterations when using 
  {\tt AMRsolve\_HG\_Prec};
\item {\tt AMRGravitySolve\_restol} \{$\ge 0.0$\; [0.0]\} -- relative
  linear residual tolerance for preconditioner system (0.0 disables
  the residual check, forcing a fixed number of iterations);
\item {\tt AMRGravitySolve\_Jaciters} \{$>0$\; [3]\} -- internal
  number of Jacobi smoothing iterations that preconditioner applies to
  the full AMR hierarchy;
\item {\tt AMRGravitySolve\_rlxtype} \{0, 1, 2, 3\; [2]\} -- choice of 
  relaxation method for inner multigrid solver within preconditioner
  (Jacobi 0, weighted Jacobi 1, symmetric red/black Gauss-Seidel 2,
  nonsymmetric red/black Gauss-Seidel 3);
\item {\tt AMRGravitySolve\_npre} \{$>0$\; [2]\} -- number of pre- and
  post-smoothing relaxation iterations applied in the inner multigrid
  solver within preconditioner. 
\end{itemize}





\subsection{{\tt AMRGravitySolve} Results}
\label{sec:AMRGravitySolve_results}

We have investigated the {\tt AMRGravitySolve} solver for both
accuracy and efficiency using a modification of the built-in \enzo
test {\tt AMRGravityTest}.  This test problem is constructed using a
dark-matter only simulation, consisting of a single very massive
particle near the center of the computational domain, surrounded by a
large number of very light, randomly placed particles.  The potential
field is then calculated using either of the built-in \enzo gravity
solvers, and the gravitational acceleration is calculated on each
particle and compared against the semi-analytical radial and
tangential forces that should arise from a $r^{-2}$ potential law.
Plots of results computed using both the original FFT-based gravity
solver and the {\tt AMRGravitySolve} solver are shown in Figure
\ref{fig:AMRGravityTest_results}. 

\begin{figure}[h]
\[
  \centerline{\myfigpng{GravityTest_orig}{2.5in}
    \myfigpng{GravityTest_new}{2.5in} }
\]
\caption{Test results using original (left) gravity solver and 
  {\tt AMRGravitySolve} (right).  Both tests use 10000 particles on a
  $64^3$ grid.  The RMS error for both tests is
  0.0755 and 0.0732, respectively, and appears to arise from reduced
  error at the coarse-fine grid boundary seen at approximately $r=4$.} 
\label{fig:AMRGravityTest_results}
\end{figure}

Similarly, we have constructed a parallel version of this test for
investigation of weak scaling, entitled 
{\tt AMRGravity\_weak\_scaling}.  This test is similar to 
{\tt AMRGravityTest}, but here each processor initializes a single
heavy particle and a large number of randomly-placed light particles
within it's root-grid processor tile.  While the test problem no
longer admits a simple semi-analytical solution, it does allow
parallel tests with an AMR hierarchy of a specified depth.  The test
does not initialize a static hierarchy, but instead initializes only
the root grid and enables adaptive mesh refinement.  We use a
refinement factor of 2 between levels, and refine at each level
geometrically to ensure that each refinement level contains the same
total number of cells as its parent level.  We ran tests of this
solver using AMR depths ranging from 1 through 9 levels, and processor
counts $p=k^3, k=1,\ldots,7$ (from 1 to 343 processors).  Each test
was performed for 8 root-grid time steps, to ensure that AMR
refinement had proceeded fully to the finest level, and timing
statistics were taken for only the final gravity solve in this time
step.  These tests were performed using both an un-preconditioned
BiCGStab linear solver for the gravity system, and a BiCGStab solver
preconditioned using the {\tt AMRsolve\_HG\_Prec} class (HG prec.).
Unfortunately, the un-preconditioned solver was unable to converge
within a maximum allowed 5000 iterations for the two deepest
hierarchies (8 and 9 levels, respectively), so results from those
tests are omitted.

In figure \ref{fig:AMRGravity_weak_scaling_results1} we plot the total
solution time versus processor count for each AMR depth, to
demonstrate the weak-scaling performance of the solvers.  As is
clearly evident in these plots, after an expected increase in solution
time when going from 1 to 8 processes, the total solution time for
each hierarchy depth remains relatively flat, increasing at the
optimal $O(log p)$ rate indicative of multigrid solvers, with the
HG-preconditioned results uniformly outperforming the
un-preconditioned solver.

\begin{figure}[h]
\[
  \centerline{\myfigpdf{scaling_np}{2.5in}
    \myfigpdf{scaling_p}{2.5in} }
\]
\caption{Weak scaling results using an un-preconditioned (left) and
  HG-preconditioned (right) gravity solver.  Curves are shown for
  differening AMR hierarchy depths, and indicate near-optimal weak
  scaling as the parallelism is increased, after an expected run-time
  increase when transitioning from serial (1 process) to parallel (8
  processes).  Additionally, the benefit from preconditioning is
  clearly evident, with curves on the right plot uniformly lower than
  corresponding curves on the left.} 
\label{fig:AMRGravity_weak_scaling_results1}
\end{figure}

Similarly, in figure \ref{fig:AMRGravity_weak_scaling_results2} we
plot the average solution time per computational cell as levels are
added to the AMR hierarchy, for a variety of different processor
counts.  The solution time increase when going from 1 to 2 levels may
be easily understood due to the sudden increase in computational
infrastructure required when going from a uniform to an adaptive
mesh.  Moreover, in these plots the benefit of the HG preconditioner
is more obvious due to the significantly flatter curves in the plot at
the right.  However, it is also clear that the efficacy of the HG
preconditioner reduces for deeper AMR hierarchies, seen here through
the increase when going deeper than 7 levels of refinement.  From
these plots, it is evident that additional research is needed to
construct a preconditioner for deep AMR runs.

\begin{figure}[h]
\[
  \centerline{\myfigpdf{timing_np}{2.5in}
    \myfigpdf{timing_p}{2.5in} }
\]
\caption{Results showing average solution per computational cell as
  the mesh is refined.  The un-preconditioned BiCGStab solver is shown
  at left, and the HG-preconditioned solver at right, with curves
  shown for runs using differing processor counts.  Note the much
  flatter curves when using the HG-preconditioner, at least through 7
  levels of mesh refinement, notwithstanding the expected increase
  when transitioning from unigrid (1 level) to AMR (2 levels).  Note:
  without preconditioning the solver could no longer converge within a
  maximum allowed 5000 iterations when using 8- and 9-level hierarchies.}
\label{fig:AMRGravity_weak_scaling_results2}
\end{figure}





\section{The {\tt AMRFLDSplit} Class}
\label{sec:AMRFLDSplit}

As with the {\tt AMRGravitySolve} routine, the {\tt AMRFLDSplit}
class implements an \enzo interface to the \\
{\tt AMRsolve\_Hypre\_FLD} class.  As with the variety of other
radiation transport solvers within \enzo, {\tt AMRFLDSplit} is called
near the beginning of {\tt EvolveLevel}, and is enabled through the
{\tt TRANSFER} compile-time option.

We note that in order to use the {\tt AMRFLDSplit} class, the
following {\tt BaryonField}s must exist within the \enzo grids:
\begin{itemize}
\item {\tt RadiationFreq0} -- contains the
  frequency-integrated radiation field,
\item {\tt Emissivity0} -- contains the emissivity field that
  acts as the source of ionizing photons for the radiation field,
\item {\tt HIDensity} -- used to calculate the opacity of
  each grid cell,
\item {\tt kphHI} -- used to temporarily hold the old radiation field
  during the radiation solve, and afterwards is used to communicate HI
  photo-ionization rates to {\tt Enzo}'s chemistry routines,
\item {\tt PhotoGamma} -- used to temporarily hold the opacity field
  during the radiation solve, and afterwards is used to communicate
  photo-heating rates to {\tt Enzo}'s cooling routines.
\end{itemize}
In other words, the {\tt AMRFLDSplit} class assumes that it will only
be used in simulations that at a minimum include Hydrogen chemistry.
Optionally, if Helium chemistry is also present, the {\tt AMRFLDSplit}
class will interact with the \enzo grids' {\tt BaryonField}s:
\begin{itemize}
\item {\tt HeIDensity} -- used in computing the opacity of each grid cell,
\item {\tt HeIIDensity} -- used in computing the opacity of each grid cell,
\item {\tt kphHeI} -- used to communicate HeI photo-ionization rates to
  {\tt Enzo}'s chemistry routines,
\item {\tt kphHeII} -- used to communicate HeI photo-ionization rates to
  {\tt Enzo}'s chemistry routines.
\end{itemize}
Lastly, since the radiation fields evolved by {\tt AMRFLDSplit} do not
include frequencies below $\nu_{HI}$, the \enzo baryon field 
{\tt kdissH2I} is always set to 0.  It is only accessed if the \enzo
input parameter {\tt MultiSpecies} is greater than 1.




\subsection{Input Parameters}
\label{sec:AMRFLDSplit_parameters}

In addition to compiling \enzo to enable the \amrsolve framework and
the \hypre linear solver library, there are a small number of
input parameters required to enable the {\tt AMRFLDSplit} class.
These are separated into two groups, one set of parameters resides in
the main \enzo parameter file, and another set may reside in a
separate FLD-specific parameter file.  First, the arguments that go in
the main \enzo parameter file are primarily used to inform \enzo in
how to to enable and use {\tt AMRFLDSplit}.  The allowed parameters
when using {\tt AMRFLDSplit} are specified in \{\}:
\begin{itemize}
\item {\tt RadiativeTransferFLD} \{2\} -- this informs \enzo to run the
  FLD module as a stand-alone solver (other configurations have it
  work with the ray-tracing solver);
\item {\tt ImplicitProblem} \{6\} -- this informs \enzo to use {\tt
  AMRFLDSplit} as opposed to the other (unigrid) FLD solvers;
\item {\tt RadiativeTransferFLDCallOnLevel} \{$\ge 0$\; [0]\} -- this
  specifies the level on which to call {\tt AMRFLDSplit} (see section
  \ref{sec:AMRFLDSplit_hierarchy}); 
\item {\tt RadiativeTransferOpticallyThinH2} \{0\} -- this enables H2
  cooling for Lyman-Werner radiation (requires {\tt
    RadiativeTransferFLD} set to 1), so should be disabled when
  running FLD as a stand-alone radiation solver;
\item {\tt MultiSpecies} \{0, 1\} -- this enables {\tt Enzo}'s chemistry
  solvers;
\item {\tt RadiativeCooling} \{0, 1\} -- this enables {\tt Enzo}'s gas
  cooling solvers;
\item {\tt RadHydroParamfile} -- this should specify a filename
  (relative to the runtime directory) that contains FLD-specific
  solver parameters (see below).  This can be the name of this same
  file, or it can specify a different file.
\end{itemize}

The second (and larger) set of parameters control the inner workings
of the {\tt AMRFLDSplit} solver.  All parameters have a recommended
default value shown in [], but may be overridden by the user: 
\begin{itemize}
\item {\tt RadHydroESpectrum} \{-1, 0, 1\; [1]\} -- choice of assumed
  radiation spectrum (see section \ref{sec:AMRFLDSplit_spectrum});
\item {\tt RadHydroModel} \{1, 4\; [1]\} -- choice of isothermal (4)
  vs temperature-dependent (1) chemistry rates;
\item {\tt RadHydroChemistry} \{1, 3\; [1]\} -- choice of
  Hydrogen+Helium (3) or Hydrogen-only (1) chemistry model;
\item {\tt RadHydroMaxDt} \{$>0$\; [$\infty$]\} -- maximum allowed radiation
  time step size (in internal, i.e.~normalized, \enzo units);
\item {\tt RadHydroMinDt} \{$\ge 0$\; [0.0]\} -- minimum allowed
  radiation time step size (in normalized units);
\item {\tt RadHydroInitDt} \{$>0$\; [$\infty$]\} -- initial radiation time
  step size (in normalized units, will be limited by main \enzo time
  step size);
\item {\tt RadHydroDtControl} \{0, 1, 2\; [2]\} -- choice of time step
  adaptivity algorithm (I is 0, PI is 1, PID is 2; see section
  \ref{sec:AMRFLDSplit_dt_selection} for further details);
\item {\tt RadHydroMaxSubcycles} \{$\ge 1.0$\; [1.0]\} -- maximum
  desired ratio between \enzo time step and radiation time step
  ($\dt / \dt_E$);
\item {\tt RadHydroDtNorm} \{$\ge 0.0$\; [2.0]\} -- choice of norm in
  estimating temporal error (see section \ref{sec:AMRFLDSplit_dt_selection});
\item {\tt RadHydroDtRadFac} \{$>0.0$\; [$\infty$]\} -- desired relative
  accuracy of radiation field (see section \ref{sec:AMRFLDSplit_dt_selection});
\item {\tt RadiationScaling} \{$>0.0$\; [1.0]\} -- unit scaling factor
  to convert between \enzo radiation units and {\tt AMRFLDSplit}
  radiation units (unnecessary if \enzo normalization sufficient, see
  section \ref{sec:AMRFLDSplit_variable_rescaling});
\item {\tt RadHydroTheta} \{$0.0 \le \theta \le 1.0$\; [1.0]\} --
  implicitness parameter within time-stepping method (0 is explicit
  Euler, 0.5 is trapezoidal, 1.0 is implicit Euler);
\item {\tt RadiationBoundaryX0Faces, RadiationBoundaryX1Faces,
  RadiationBoundaryX2Faces} \{0, 1, 2\; [0 0]\} -- integer pairs
  specifying boundary condition type at each face (0 periodic, 1
  Dirichlet, 2 Neumann), zero-valued fields are set by default but may
  be overridden by {\tt ProblemType}-specific calls within {\tt
    AMRFLDSplit\_Initialize.C};
\item {\tt RadHydroSolType} \{0, 1, 2, 3, 4\; [1]\} -- choice of
  \hypre algorithm for the main FLD linear system (0 FAC, 1 BiCGStab,
  2 BiCGStab with BoomerAMG preconditioner, 3 GMRES, 4 PFMG),
  currently only 1 and 3 are known to work for AMR simulations, while
  4 is recommended for unigrid simulations;
\item {\tt RadHydroSolTolerance} \{1e-15 $\le$ tol $\le$ 1.0\;
  [1e-8]\} -- relative tolerance for FLD linear system residual;
\item {\tt RadHydroMaxMGIters} \{$>0$\; [200]\} -- number of allowed
  iterations in the main \hypre linear solver;
\item {\tt RadHydroMGRelaxType} \{0, 1, 2, 3\; [1]\} -- choice of
  relaxation method if using {\tt RadHydroSolType} of 4 (Jacobi 0,
  weighted Jacobi 1, symmetric red/black Gauss-Seidel 2, nonsymmetric
  red/black Gauss-Seidel 3);
\item {\tt RadHydroMGPreRelax} \{$>0$\; [1]\} -- number of pre-relaxation
  sweeps when using {\tt RadHydroSolType} of 4;
\item {\tt RadHydroMGPostRelax} \{$>0$\; [1]\} --number of
  post-relaxation sweeps when using {\tt RadHydroSolType} of 4;
\item {\tt RadHydroSolPrec} \{0, 1\; [1]\} -- flag to enable the {\tt
  AMRsolve\_HG\_Prec} class to precondition the \hypre solver (only
  available for {\tt RadHydroSolType} 1 and 3);
\item {\tt RadHydroSol\_precmaxit} \{$>0$\; [1]\} -- maximum number of
  preconditioner sweeps per outer solver iteration;
\item {\tt RadHydroSol\_precnpre} \{$>0$\; [1]\} -- number of
  pre-relaxation sweeps per preconditioner iteration;
\item {\tt RadHydroSol\_precnpost} \{$>0$\; [1]\} -- number of
  post-relaxation sweeps per preconditioner iteration;
\item {\tt RadHydroSol\_precJacit} \{$>0$\; [2]\} -- number of Jacobi
  iterations on the full AMR hierarchy per preconditioner iteration;
\item {\tt RadHydroSol\_precrelax} \{0, 1, 2, 3\; [1]\} -- relaxation
  type for root-grid PFMG solve within the preconditioner (same values
  as listed for {\tt RadHydroMGRelaxType} above;
\item {\tt RadHydroSol\_precrestol} \{$\ge 0.0$\; [0.0]\} -- relative
  linear residual tolerance for preconditioner system (0.0 disables
  the residual check, forcing a fixed number of iterations).
\end{itemize}





\subsection{Flux-limited diffusion radiation model}
\label{sec:rad_model}

We begin with the equation for flux-limited diffusive radiative
transfer in a cosmological medium \cite{ReynoldsHayesPaschosNorman2009},
\begin{equation}
\label{eq:radiation_PDE}
  \partial_{t} E + \frac1a \nabla\cdot\(E\vb\) =
    \nabla\cdot\(D\,\nabla E\) - \frac{\dot{a}}{a} E - c\kappa E + \eta,
\end{equation}
where here the comoving radiation energy density $E$, emissivity
$\eta$ and opacity $\kappa$ are functions of space and time.  In this
equation, the frequency-dependence of the radiation energy has been
integrated out, under the premise of an assumed radiation energy
spectrum (see section \ref{sec:AMRFLDSplit_spectrum}).

The function $D$ in the above equation \eqref{eq:radiation_PDE} is
the {\em flux-limiter} that depends on $E$, $\nabla E$ and the 
opacity $\kappa$,  
\[
   D = \left[\begin{array}{ccc} D^{1} & & \\ & D^{2} & \\ & & D^{3} \end{array}\right],
\]
where the directional limiters $D^i$ are given by \cite{Morel2000}
\begin{align}
  \label{eq:Larsen_limiter}
   D^{i}\left(E,\tfrac{\partial E}{\partial x_i},\kappa\right) = 
     c \left[ 9\kappa^2 + 
     \left(R\left(E,\tfrac{\partial E}{\partial x_i}\right)\right)^2\right]^{-1/2}, \qquad
   R^i\left(E,\tfrac{\partial E}{\partial x_i}\right) = 
     \max\left\{\frac{\left|\tfrac{\partial E}{\partial x_i} \right|}{E}, 10^{-20} \right\}. 
\end{align}
Within each uniform grid patch, the Laplace operator in
\eqref{eq:radiation_PDE} is discretized using second-order centered
differences on a standard cell-centered finite-volume mesh, with the
flux limiters computed at cell faces, i.e.~in the $x_1$ direction we
use 
\begin{align*}
  &\left[\frac{\partial}{\partial x_1}\left(D_1
    \frac{\partial E}{\partial x_1}\right)\right]_{i,j,k} \ \approx \ 
  \frac{1}{\dx_1} \left[
    D^1_{i+1/2,j,k}\frac{E_{i+1,j,k}-E_{i,j,k}}{\dx_1} -
    D^1_{i-1/2,j,k}\frac{E_{i,j,k}-E_{i-1,j,k}}{\dx_1} \right],\\
  \text{where, e.g.} & \\
  & D^1_{i+1/2,j,k} = D^{i}\left(
     \tfrac{E_{i+1,j,k}+E_{i,j,k}}{2},
     \tfrac{E_{i+1,j,k}-E_{i,j,k}}{\dx_1},
     \tfrac{\kappa_{i+1,j,k}+\kappa_{i,j,k}}{2}\right).
\end{align*}
At coarse/fine grid boundaries, we use piecewise constant
interpolation to shift dependence from non-existent same-grid
neighbors to the coarser/finer neighbor across the level boundary,
and adjust the relevant differencing parameters $\dx_i$
appropriately. 



\subsection{Radiation Spectrum}
\label{sec:AMRFLDSplit_spectrum}

As mentioned in section \ref{sec:rad_model}, the frequency-dependence
of the radiation field has been integrated out, under the premise of
an assumed radiation energy spectrum.  Assuming the frequency, space
and time-dependent radiation field is denoted $E_{\nu}(\nu,\xvec,t)$,
we first assume the radiation field exhibits the same frequency
spectrum throughout the spatial domain and time evolution of the
simulation, denoted $\chi(\nu)$, i.e.
\begin{align}
  \label{eq:spectrum}
  E_{\nu}(\nu,\xvec,t) = \tilde{E}(\xvec,t) \chi(\nu),
\end{align}
where $\tilde{E}$ is an intermediate quantity (for analysis) that is
never computed.  We then form a so-called ``grey'' model through
integration over frequency space, i.e.
\begin{align}
  \label{eq:greyFLD}
  E(\xvec,t) = \int_{\nu_{HI}}^{\infty} E_{\nu}(\nu,\xvec,t)\,\mathrm{d}\nu 
    = \tilde{E}(\xvec,t) \int_{\nu_{HI}}^{\infty} \chi(\nu)\,\mathrm{d}\nu,
\end{align}
We note that if the assumed spectrum is the Dirac delta function,
$\chi(\nu) = \delta_{\nu_{HI}}(\nu)$, $E$ is a monochromatic radiation
energy density at the ionization threshold of HI, and the
$\left(-\frac{\dot{a}}{a}E\right)$ term (obtained through integration
by parts of the redshift term
$\left(\frac{\dot{a}}{a}\partial_{\nu}E_{\nu}\right)$ present in the
corresponding frequency-dependent equation) is omitted from
\eqref{eq:radiation_PDE}.

The {\tt AMRFLDSplit} solver allows for user control over this assumed
spectrum, $\chi(\nu)$.  We currently have three pre-defined options in
the code, although more can easily be added within the function \\
{\tt AMRFLDSplit\_RadiationSpectrum.C}.  The three current options may
be selected through the input parameter {\tt RadHydroESpectrum}
discussed above, and consist of:
\begin{itemize}
\item[(-1)] monochromatic spectrum at frequency $h\nu_{HI} = 13.6$ eV, i.e.
  $\chi(\nu) = \delta_{\nu_{HI}}(\nu)$.  
\item[(0)] power law spectrum,
  \[
  \chi(\nu) = \left(\frac{\nu}{\nu_{HI}}\right)^{-1.5}
  \]
\item[(1)] $T=10^5$ K blackbody spectrum, 
  \[
  \chi(\nu) = \frac{8 \pi h
    \left(\frac{\nu}{c}\right)^3}{\exp\left(\frac{h\nu}{k_b 10^5}\right)-1}.
  \]
\end{itemize}
Additional spectra may be easily added by modifying the file 
{\tt AMRFLDSplit\_RadiationSpectrum.C} to implement a new 
{\tt RadHydroESpectrum}-dependent function $\chi(\nu)$, along with use
of the new value for {\tt RadHydroESpectrum} in a user's FLD parameter
file.  The units of this function do not matter, since we assume
$\chi(\nu)$ to be dimensionless.  At the beginning of a simulation,
the specified spectrum in {\tt AMRFLDSplit\_RadiationSpectrum.C} is
integrated numerically in the routine 
{\tt AMRFLDSplit\_ComputeRadiationIntegrals.C} to form high-accuracy
approximations of
\begin{align*}
  \int_{\nu_{HI}}^{\infty} \chi(\nu)\, \mathrm{d}\nu, \qquad
  \int_{\nu_i}^{\infty} \sigma_i(\nu)\, \chi(\nu)\, \mathrm{d}\nu,\qquad
  \int_{\nu_{i}}^{\infty} \frac{\sigma_{i}(\nu)\,\chi(\nu)}{\nu}\, \mathrm{d}\nu,
  \qquad i = \{\text{HI}, \text{HeI}, \text{HeII}\},
\end{align*}
where we denote $\nu_{i}$ as the ionization threshold of the species
$\mn_i$.  These integrals are then combined at each time step of the
simulation to form all of the spectrum-related integrals discussed in
the remainder of this section.

We do note that if a user wishes to modify the {\tt AMRFLDSplit}
solver to account for other types of radiation with lower frequencies
than $\nu_{HI}$ (e.g.~Ly-$\alpha$), then in addition to modifying how
the spectra are defined in {\tt AMRFLDSplit\_RadiationSpectrum.C},
they will also need to modify the limits of integration used within
the routine {\tt AMRFLDSplit\_ComputeRadiationIntegrals.C}.

The choice of spectrum affects the interactions between the radiation
field and the chemistry.  Specifically, since gas photo-heating
rate $G$ for a Hydrogen+Helium simulation depends on radiation through
the equation 
\[
  G = \sum_{i} \frac{c\,\mn_{i}}{\rhob}
    \int_{\nu_{i}}^{\infty} \sigma_{i}(\nu)\, E_{\nu}(\nu)
    \left(1-\frac{\nu_{i}}{\nu}\right)\, \mathrm{d}\nu, 
    \qquad i = \{\text{HI}, \text{HeI}, \text{HeII}\},
\]
then under assumption \eqref{eq:spectrum} this becomes
\begin{equation}
  \label{eq:G_nLTE}
  G = \sum_{i} \frac{c\,\mn_{i}\,E}{\rhob} 
    \left[\int_{\nu_i}^{\infty} \sigma_i(\nu)\, \chi(\nu)
    \left(1-\frac{\nu_i}{\nu}\right)\, \mathrm{d}\nu\right] \bigg/
    \left[\int_{\nu_{HI}}^{\infty} \chi(\nu)\, \mathrm{d}\nu\right],
    \qquad i = \{\text{HI}, \text{HeI}, \text{HeII}\}.
\end{equation}

Similarly, the photo-ionization rates resulting from a
frequency-dependent radiation field may be written
\[
  \Gamma_i^{ph} = \int_{\nu_{i}}^{\infty} \frac{c\, \sigma_{i}(\nu)\,
    E_{\nu}}{h\nu}\, \mathrm{d}\nu,
    \qquad i = \{\text{HI}, \text{HeI}, \text{HeII}\}.
\]
Again, under the assumption of a fixed frequency spectrum
\eqref{eq:spectrum} this simplifies to
\begin{equation}
\label{eq:Gamma}
  \Gamma_i^{ph} = \frac{c\,E}{h}
  \left[\int_{\nu_{i}}^{\infty} \frac{\sigma_{i}(\nu)\,\chi(\nu)}{\nu}\,
    \mathrm{d}\nu\right] \bigg/
    \left[\int_{\nu_{HI}}^{\infty} \chi(\nu)\, \mathrm{d}\nu\right],
    \qquad i = \{\text{HI}, \text{HeI}, \text{HeII}\}.
\end{equation}
The photo-heating and photo-ionization rates are set within the file
{\tt AMRFLDSplit\_FillRates.C}, to eventually be passed on to 
{\tt Enzo}'s routines for chemistry and cooling.  We note that if a
user wishes to modify {\tt AMRFLDSplit} to perform alternate types of
radiation transport, and if their new model interacts with the
chemistry and cooling differently than the model described here, these
modeling modifications should be incorporated into the {\tt
  AMRFLDSplit\_FillRates.C} routine.

Moreover, a frequency-dependent opacity $\kappa_{\nu}$ of a
Hydrogen+Helium medium to radiation transport may be given by the
formula
\[
  \kappa_{\nu} = \sum_i \sigma_i(\nu)\, \mn_i, 
  \qquad i = \{\text{HI}, \text{HeI}, \text{HeII}\}.
\]
However, for simulation of our integrated radiation equation
\eqref{eq:radiation_PDE}, we need the term
\[
   c\,\kappa\,E = c\int_{\nu_{HI}}^{\infty} \kappa_{\nu}
   E_{\nu}\,\mathrm{d}\nu.
\]
Relating these formulas to the assumption \eqref{eq:spectrum}, we
may compute the integrated opacity due to Hydrogen+Helium as
\begin{equation}
\label{eq:opacity}
  \kappa \ = \ \sum_i
  \left[\mn_{i} \int_{\nu_{i}}^{\infty}
    \sigma_{i}(\nu)\,\chi(\nu)\,\mathrm{d}\nu\right] \bigg/
  \left[\int_{\nu_{HI}}^{\infty} \chi(\nu)\,\mathrm{d}\nu\right].
\end{equation}

The final component that must be accounted for in using a
frequency-integrated radiation field is the emissivity.  Given a
frequency-, space- and time-dependent emissivity function of the form
$\eta_{\nu}(\nu,\xvec,t)$, we must source the FLD radiation equation
\eqref{eq:radiation_PDE} using a space- and time-dependent emissivity
field 
\begin{equation}
\label{eq:emissivity}
  \eta(\xvec,t) =
  \int_{\nu_{HI}}^{\infty}\eta_{\nu}(\nu,\xvec,t)\,\mathrm{d}\nu. 
\end{equation}
The {\tt AMRFLDSplit} solver currently implements a wide variety of
{\tt ProblemType}-dependent emissivity fields in the function 
{\tt AMRFLDSplit\_RadiationSource.C}.  Additional user-specified 
{\tt ProblemType}-dependent emissivity functions may be easily 
inserted into this file.  We assume that this integrated emissivity is
specified in CGS units of erg cm$^{-3}$ s$^{-1}$, and perform internal
rescaling to convert this to code units within the solver.




\subsection{Implicit Solution Approach}
\label{sec:AMRFLDSplit_rad_solve}

Since we treat the radiation as a density field the advective portion
of \eqref{eq:radiation_PDE}, i.e. 
\[
  \partial_{t} E = -\frac1a \nabla\cdot\(E\vb\),
\]
is evolved using {\tt Enzo}'s hydrodynamics routines.  The 
{\tt AMRFLDSplit} solver therefore focuses on evolution of the
remainder of \eqref{eq:radiation_PDE}, i.e.
\[
  \partial_{t} E = \nabla\cdot\(D\,\nabla E\) - \frac{\dot{a}}{a} E - c\kappa E + \eta.
\]
We lag the nonlinearity in the limiter to the previous time step, and
then discretize the remainder of the equation in time using a standard
two-level $\theta$-method,
\begin{align}
  \label{eq:radiation_PDE_theta}
  E^n - E^{n-1} &- \theta\dt\left(\nabla\cdot\(D^{n-1}\,\nabla E^n\) - \frac{\dot{a}}{a} E^n -
    c\kappa^n E^n + \eta^n\right) \\ 
  \notag
  & - (1-\theta)\dt\left(\nabla\cdot\(D^{n-1}\,\nabla E^{n-1}\) - \frac{\dot{a}}{a} E^{n-1} -
    c\kappa^{n-1} E^{n-1} + \eta^{n-1}\right) = 0,
\end{align}
where $0\le\theta\le 1$ defines the time-discretization.  As discussed
in section \ref{sec:AMRFLDSplit_parameters}, the value of $\theta$ may
be specified by the user in the FLD parameter input file. Recommended
values of $\theta$ are 1.0 (implicit Euler -- $O(\dt)$ accurate,
$A$ and $L$ stable) and $0.5$ (trapezoidal -- $O(\dt^2)$
accurate, $A$ stable).  However, any nonzero value of $\theta$ will
result in a linearly-implicit system of equations for the
time-evolved radiation energy density $E^n$.  We write this in
predictor-corrector form as
\begin{align}
\label{eq:linear_system}
  J s = b, \quad\text{where}\quad s = E^n - E^{n-1}.
\end{align}
We then approximately solve this linear equation for the update $s$,
to a relative residual tolerance $\delta$,
\begin{align}
\label{eq:linear_system_approx}
  \| J s - b \|_2 \le \delta \|b\|_2.
\end{align}
As discussed in section \ref{sec:AMRFLDSplit_parameters}, the
tolerance $\delta$ may be specified through the input parameter \\
{\tt RadHydroSolTolerance}, and control over the solver algorithm and
its parameters may be controlled through a myriad of input parameters.




\subsection{Time-step selection}
\label{sec:AMRFLDSplit_dt_selection}

Time steps are chosen adaptively in an attempt to control error in the
calculated solution.  To this end, we first define an heuristic
measure of the time accuracy error in the radiation field $E(t^n)$ as
either
\begin{align}
\label{eq:time_error}
  \epsilon^{(n)} = \left(\frac{1}{V} \sum_{i} \Delta V_i 
    \left(\frac{E_i^{n}-E_i^{n-1}}{\omega_i}\right)^p\right)^{1/p}, 
\end{align}
or
\begin{align}
\label{eq:time_error_inf}
  \epsilon^{(n)} = \max_i\left|\frac{E_i^{n}-E_i^{n-1}}{\omega_i}\right|,
  \quad\text{if}\; p=0.0,
\end{align}
where the weights $\omega_i$ are given by
\begin{align}
\label{eq:time_weighting}
  \omega_i &= \sqrt{\left|E_i^n E_i^{n-1}\right|} + 10^{-1},
\end{align}
i.e.~we scale the radiation by the geometric mean of the old and new
states, adding on a floor value of $10^{-1}$ in case any of the states
are too close to zero.  Moreover, we omit from this sum or max all
coarse-grid cells that are covered by a finer grid patch.  In the
$p$-norm estimate, $\Delta V_i$ corresponds to the normalized volume
of the grid cell containing $E_i$. This approach works well when the
internal solution variables are unit-normalized, or at least close to
unit-normalized, since the difference between the old and new
solutions, divided by this weighting factor $\omega_i$, should give a
reasonable estimate of the number of significant digits that have
changed between time steps. 

With these error estimates
\eqref{eq:time_error}-\eqref{eq:time_weighting}, we set the new
radiation time step size based on a user-input relative tolerance
$\tau_{\text{tol}}$ in an attempt to minimize the requisite number of
time steps to achieve the desired temporal accuracy.  Since $err$ is
an estimate of the relative error, and $\tau_{\text{tol}}$ is the
target relative error, we compute the ratio of these at each time
step,
\begin{align}
  \label{eq:time_accuracy}
  \zeta^{(n)} &= \max\left\{\frac{\epsilon^{(n)}}{\tau_{\text{tol}}},\; 10^{-8} \right\}.
\end{align}
With this estimate of our achieved accuracy in each time step, we
allow the user to choose between three time adaptivity control
algorithms \cite{KennedyCarpenter2003}:
\begin{itemize}
\item[(0)] I controller: 
  \begin{align}
    \label{eq:time_estimateI}
    \dt_E^{(n+1)} = \dt_E^{(n)}\, \(\zeta^{(n)}\)^{-1/q}
  \end{align}
\item[(1)] PI controller:
  \begin{align}
    \label{eq:time_estimatePI}
    \dt_E^{(n+1)} = \dt_E^{(n)}\, \(\zeta^{(n)}\)^{-0.7/q}\, \(\zeta^{(n-1)}\)^{0.4/q}
  \end{align}
\item[(2)] PID controller:
  \begin{align}
    \label{eq:time_estimatePID}
    \dt_E^{(n+1)} = \dt_E^{(n)}\, \(\zeta^{(n)}\)^{-0.49/q}\, \(\zeta^{(n-1)}\)^{0.34/q}\, \(\zeta^{(n-2)}\)^{-0.1/q},
  \end{align}
\end{itemize}
where in each of these $q$ equals 2 for $\theta=0.5$ and $q=1$
otherwise, matching the order of accuracy of each method.  In the
above formulas, $\dt_E$ represents the time step size used in evolving
the radiation field via {\tt AMRFLDSplit}.  The values of $p$,
$\tau_{\text{tol}}$ and the time adaptivity control algorithm may be
specified using the FLD parameter file inputs {\tt RadHydroDtNorm} 
{\tt RadHydroDtRadFac} and {\tt RadHydroDtControl}, respectively.

Since $E$ may evolve at a faster time scale than the remaining
components in \enzo, we allow time subcycling of the radiation field.
Denoting the radiation time step as $\dt_{E}$ and the time step size
for the remaining \enzo fields as $\dt$, then we use the input
parameter {\tt RadHydroMaxSubcycles} to enforce the limit
\[
  \dt \ \le \ \text{\tt RadHydroMaxSubcycles} * \dt_E,
\]
where $\dt_E$ results from our time adaptivity algorithm.  

Finally, a user may specify minimum and maximum values for $\dt_E$
through the input parameters {\tt RadHydroMinDt} and 
{\tt RadHydroMaxDt}.  These values should be supplied in {\tt Enzo}'s
scaled time units.


{\bf Notes/Warnings:}
\begin{itemize}
\item The radiation time adaptivity algorithm may be disabled by
  leaving the input parameter {\tt RadHydroDtRadFac} unset (i.e.~set
  to $\infty$).
\item Since the FLD solve requires coupling throughout the entire
  computational domain, the {\tt AMRFLDSplit} module may comprise a
  large fraction of the total run time for a simulation, though this 
  cost can be reduced by either setting a looser (larger) tolerance
  {\tt RadHydroDtRadFac} or a larger subcycling factor {\tt
    RadHydroMaxSubcycles}. 
\item For problems with significant interaction between radiation and
  chemistry or cooling, it is {\em strongly} recommended that the
  tolerance be set to a maximum of $10^{-2}$ and that subcycling be
  disabled (by leaving {\tt RadHydroMaxSubcycles} at 1.0).  This is
  because the chemistry and cooling routines are handled by separate
  \enzo modules, but these physical processes are tightly coupled.
  Hence, by loosening the coupling in the solvers the physics will not
  be properly modeled.
\end{itemize}


We further note that when run in combination with Enzo's hydrodynamics
routines, both modules will limit their maximum time step sizes to the
minimum of $\dt_{\text{E}}$ and $\dt_{\text{CFL}}$.  As a result, in
some physical regimes, the global time step size will be limited based
on the radiation time scale, and in other regimes it will be limited
by the hydrodynamic time scale. 




\subsection{Variable rescaling}
\label{sec:AMRFLDSplit_variable_rescaling}

In the case that {\tt Enzo}'s standard unit non-dimensionalization
using {\tt DensityUnits}, {\tt LengthUnits} and {\tt TimeUnits} is
insufficient to rescale the resulting internal values for $E$ to have
nearly unit magnitude, the user may input an additional variable
scaling factor to be used inside the {\tt AMRFLDSplit} module, through
the FLD input parameter {\tt RadiationScaling}.  Denoting this
user-input value as $s_E$, we define the rescaled variable $\tilde{E}
= E / s_E$, which {\tt AMRFLDSplit} in its internal routines instead
of Enzo's internal variable $E$.  If the user does not know an
appropriate value for this scaling factor {\em a-priori}, a
generally-applicable rule of thumb is to first run the simulation for
a small number of time steps and investigate {\tt Enzo}'s HDF5 output
files to see the magnitude of the values stored internally by \enzo;
if these are far from unit-magnitude, a scaling factor should be used. 




\subsection{Boundary conditions}
\label{sec:AMRFLDSplit_boundary_conditions}

As the radiation equation \eqref{eq:radiation_PDE} is parabolic,
boundary conditions must be supplied on the radiation field $E$.  The
{\tt AMRFLDSplit} module allows three types of boundary conditions to
be placed on the radiation field:
\begin{itemize}
\item[(0)] Periodic,
\item[(1)] Dirichlet, i.e.~$E(x,t) = g(x), \; x\in\partial\Omega$, and
\item[(2)] Neumann, i.e.~$\nabla E(x,t)\cdot n = g(x), \; x\in\partial\Omega$.
\end{itemize}
In most cases, the boundary condition types (and values of $g$) are
problem-dependent.  When adding new problem types, these 
{\tt ProblemType}-dependent conditions should be set near the bottom
of the file {\tt AMRFLDSplit\_Initialize.C}, otherwise these will
default to either (a) periodic, or (b) will use $g=0$, depending on
the user input boundary condition type (specified through the FLD
input parameters \\
{\tt RadiationBoundaryX0Faces}, {\tt RadiationBoundaryX1Faces}, and
{\tt RadiationBoundaryX2Faces}. 




\subsection{Refinement Strategies}
\label{sec:AMRFLDSplit_refinement}

For testing purposes, we have implemented four mesh adaptivity
refinement criteria for use with the {\tt AMRFLDSplit} solver, all of
which require the \enzo configuration option {\tt TRANSFER} enabled
for compilation.  The first three of these may be physically relevant
within astrophysical simulations, while the fourth exists purely for
verification and scaling tests.  
\begin{itemize}
\item {\tt Grid\_FlagCellsToBeRefinedByRadiationGradient.C} -- this
  adaptivity strategy is enabed with the input parameter 
  {\tt CellFlaggingMethod} set to 19.  This routine iterates over all
  of the cells in each grid, and computes a normalized gradient of the
  radiation field,
  \begin{align*}
     g_{0,i,j,k} &= \frac{E_{i+1,j,k}-E_{i-1,j,k}}{\max\{|E_{i,j,k}|, 0.1\}}\\
     g_{1,i,j,k} &= \frac{E_{i,j+1,k}-E_{i,j-1,k}}{\max\{|E_{i,j,k}|, 0.1\}}\\
     g_{2,i,j,k} &= \frac{E_{i,j,k+1}-E_{i,j,k-1}}{\max\{|E_{i,j,k}|, 0.1\}}\\
     G_{i,j,k} &= \( g_{0,i,j,k}^2 + g_{1,i,j,k}^2 + g_{2,i,j,k}^2 \)^{1/2},
  \end{align*}
  and compares this against a user-specified minimum slope for
  refinement, $r_{min}$, to determine whether the cell should be
  refined, i.e.
  \[
     G_{i,j,k} > r_{min} \;\Rightarrow\;\text{refine}, \qquad
     G_{i,j,k} \le r_{min} \;\Rightarrow\;\text{do not refine}.
  \]
  The value of $r_{min}$ is taken from the first entry in the 
  {\tt MinimumSlopeForRefinement} input array.  From heuristic tests,
  we find that a value of 1.3 for this parameter is typically
  sufficient.
\item {\tt Grid\_FlagCellsToBeRefinedByOpacity.C} -- this adaptivity
  strategy is enabed with the input parameter {\tt CellFlaggingMethod}
  set to 20.  This routine performs the same gradient-based comparisons as 
  the above routine, except that it examines gradients in the opacity
  field, $\kappa$, defined in \eqref{eq:opacity}.  The value of
  $r_{min}$ for this test is taken from the second entry in the 
  {\tt MinimumSlopeForRefinement} input array.  Heuristic tests have
  demonstrated that a reliable value for this threshold is 1.0.
\item {\tt Grid\_FlagCellsToBeRefinedByIonizedFraction.C} -- this
  adaptivity strategy is enabled with the input parameter 
  {\tt CellFlaggingMethod} set to 21.  This routine again performs the
  same gradient-based comparisons as the above routines, except that
  it examines the gradients of the ionized fraction
  \[
     \mathcal I = 1 - \frac{HI + HeI + \frac12 HeII}{\rho}.
  \]
  Here, the value of $r_{min}$ is taken from the third entry in the
  {\tt MinimumSlopeForRefinement} input array, with heuristic tests
  showing that a value of 1.25 is typically sufficient.
\item {\tt Grid\_FlagCellsToBeRefinedForWeakScaling.C} -- this
  adaptivity strategy is enabed with the input parameter 
  {\tt CellFlaggingMethod} set to 22, and should only be used for
  weak-scaling performance tests.  The routine flags cells based on
  their location in relation to the root grid tile below each cell.
  If the center of this root grid tile is located at $(c_0, c_1, c_2)$
  and has size $L_0\times L_1\times L_2)$, we set a refinement region
  on the level $l$ as 
  \[
     R_l =  \left[c_0-\frac{L_0}{2^{l+2}}, c_0+\frac{L_0}{2^{l+2}} \right] 
     \times \left[c_1-\frac{L_1}{2^{l+2}}, c_1+\frac{L_1}{2^{l+2}} \right] 
     \times \left[c_2-\frac{L_2}{2^{l+2}}, c_2+\frac{L_2}{2^{l+2}} \right].
  \]
  If the center of a given grid cell within level $l$ lies in the
  region $R_l$, it is marked for refinement. 
\end{itemize}


\subsection{FLD Hierarchy}
\label{sec:AMRFLDSplit_hierarchy}

The {\tt AMRFLDSplit} solver is designed to be called on a given level
of the hierarchy (typically the root grid level), as specified by the
main \enzo input parameter {\tt RadiativeTransferFLDCallOnLevel}.
When {\tt AMRFLDSplit} is called on a specified level, it constructs
and solves the equation \eqref{eq:radiation_PDE} on a hierarchy
spanning from the calling level to the highest level in the AMR
hierarchy.  For example, in an AMR run with 7 levels, if 
{\tt AMRFLDSplit} is called on level 2, it will form the radiation
equations on a hierarchical grid comprised of levels 2 through 7.

Since equation \eqref{eq:radiation_PDE} is parabolic, the full
implicit system is solved simultaneously, using a fixed time step in
the hierarchical solve, where the time step size is limited only by
the main \enzo time step size on the call level.  As a result, within
subcycled time steps of the {\tt EvolveLevel} routine on finer levels
of the hierarchy, the radiation field will have already been evolved
further in time, and is therefore held fixed until all of the levels
have caught back up with the current radiation field.

Furthermore, since all of the levels between the calling level and the
finest AMR level are solved simultaneously, the solution on all coarse
grid cells that are covered by a fine grid patch is over-written by
the average value of the covering cells.  A direct consequence of this
is that it is only necessary to specify the emissivity $\eta$ in cells
on the finest grid level in a given location, since coarser cells will
not contribute to the final solution.




\subsection{{\tt AMRFLDSplit} Test Problems}
\label{sec:AMRFLDSplit_test_problems}

We have set up a number of test problems that utilize the 
{\tt AMRFLDSplit} solver.  These generally mirror the existing test
problems used in the {\tt gFLDProblem} and {\tt gFLDSplit} solver
modules, though here are set to use either statically-nested or
adaptive mesh refinement.  All of these test problems are included in
the \enzo distribution, in the source directory 
{\tt run/RadiationTransportFLD/}. 
\begin{itemize}
\item {\tt RHIonization1\_AMRstatic} -- This implements ``test 1''
  from \cite{IlievEtAl2006}; a uniform density, isothermal, Hydrogen
  I-front propagation test, using a monochromatic radiation energy
  density at $h\nu$ = 13.6 eV.  This test runs the problem using a
  coarse $16^3$ root grid with 2 levels of geometric mesh
  refinement ({\tt CellFlaggingMethod} 22).
\item {\tt RHIonization1\_AMRdynamic} -- This mimics the previous
  test, but uses an adaptive mesh based the opacity gradient
  (i.e.~{\tt CellFlaggingMethod} 20).
\item {\tt RHIonization2\_AMRstatic} -- This implements ``test 2''
  from \cite{IlievEtAl2006}; a uniform density, temperature-dependent,
  Hydrogen I-front propagation test, using $T=10^5$ K blackbody
  radiation energy spectrum.  This test runs the problem using a 
  coarse $16^3$ root grid with 2 levels of geometric mesh
  refinement.
\item {\tt RHIonization2\_AMRdynamic} -- This mimics the previous
  test, but uses an adaptive mesh based on the gradient of the ionized
  fraction ({\tt CellFlaggingMethod} 21).
\item {\tt CosmoIonization\_qXzY\_AMRstatic} -- These implement ``test 
  4.6'' from \cite{ReynoldsHayesPaschosNorman2009}; cosmological
  uniform density, isothermal, Hydrogen I-front propagation tests,
  using a monochromatic radiation energy density at $h\nu$ = 13.6 eV.
  Here, the ``X'' in the test name is one of 0.5 or 0.05,
  corresponding to the cosmological deceleration parameter used in the
  test.  Similarly, ``Y'' is one of 4 or 10, corresponding to the
  initial redshift used in the run.  These tests run using a coarse
  $16^3$ root grid with 2 levels of geometric mesh refinement.
\item {\tt CosmoIonization\_qXzY\_AMRdynamic} -- These mimic the
  previous set of tests, but use an adaptive mesh based on the opacity
  gradient.
\item {\tt AMRFLDSplit\_weak\_scaling} -- This implement the 
  {\tt RHIonization2\_AMRdynamic} test, but now mirrored within each
  root-grid tile.  It is designed to perform timing tests of 
  {\tt AMRFLDSplit} as the number of MPI tasks and the number of AMR
  levels are increased.
\end{itemize}





\section{Concluding remarks}
\label{sec:conclusions}

We wish to remark that \amrsolve is designed for flexibility, to be
easily adapted to new physical models and problems.  While we have
strived to ensure that \amrsolve, {\tt AMRGravitySolve} and 
{\tt AMRFLDSplit} are bug-free, we welcome any feedback and/or
collaboration from the \enzo community, including bugfixes, usability
suggestions, requests for new physical functionality.


\bibliography{sources}
\bibliographystyle{siam}
\end{document}
