\documentclass[10pt]{article}
\usepackage{geometry}   % See geometry.pdf to learn the layout
                        % options.  There are lots.
\geometry{letterpaper}
\usepackage[latin1]{inputenc}
\usepackage{graphicx}
\usepackage{epstopdf}
\usepackage{color}
\usepackage{amsmath,amsfonts,amssymb}

\DeclareGraphicsRule{.tif}{png}{.png}{`convert #1 `dirname #1`/`basename #1 .tif`.png}

\author{Daniel R. Reynolds}
\title{Constructing the finite-volume equations for cosmological FLD}

\renewcommand{\(}{\left(}
\renewcommand{\)}{\right)}
\newcommand{\vb}{{\bf v}_b}
\newcommand{\xvec}{{\bf x}}
\newcommand{\rvec}{{\bf r}}
\newcommand{\Omegabar}{\bar{\Omega}}
\newcommand{\adot}{\dot{a}}
\newcommand{\rhob}{\rho_b}
\newcommand{\dt}{\Delta t}
\newcommand{\Enu}{E_{\nu}}
\newcommand{\Fnu}{{\bf F}_{\nu}}
\newcommand{\Pnu}{\overline{\bf P}_{\nu}}
\newcommand{\R}{\mathbb{R}}
\newcommand{\Rthree}{\R^3}
\newcommand{\eh}{e_h}
\newcommand{\ec}{e_c}
\newcommand{\Edd}{\mathcal F}
\newcommand{\Eddnu}{\Edd_{\nu}}
\newcommand{\mn}{{\tt n}}
\newcommand{\mB}{\mathcal B}
\newcommand{\mC}{{\mathcal C}}
\newcommand{\mL}{{\mathcal L}}
\newcommand{\mD}{{\mathcal D}}
\newcommand{\mDnu}{\mD_{\nu}}
\newcommand{\mCnu}{\mC_{\nu}}
\newcommand{\mLnu}{{\mathcal L}_{\nu}}
\newcommand{\mCe}{\mC_e}
\newcommand{\mLe}{\mL_e}
\newcommand{\mCn}{\mC_{\mn}}
\newcommand{\mLn}{\mL_{\mn}}
\newcommand{\Aunit}{a_{\text{unit}}}
\newcommand{\Lunit}{l_{\text{unit}}}
\newcommand{\Dunit}{\rho_{\text{unit}}}
\newcommand{\Tunit}{t_{\text{unit}}}
\newcommand{\Vunit}{v_{\text{unit}}}
\newcommand{\Eunit}{E_{\text{unit}}}
\newcommand{\Kunit}{\kappa_{\text{unit}}}
\newcommand{\tK}{\tilde{\kappa}}
\newcommand{\tT}{\tilde{t}}
\newcommand{\tX}{\tilde{x}}
\newcommand{\tY}{\tilde{y}}
\newcommand{\tZ}{\tilde{z}}
\newcommand{\tE}{\tilde{E}}
\newcommand{\tRho}{\tilde{\rhob}}
\newcommand{\tV}{\tilde{\vb}}
\newcommand{\tA}{\tilde{a}}
\newcommand{\tAdot}{\tilde{\adot}}
\newcommand{\tD}{\tilde{D}}
\newcommand{\tmD}{\tilde{\mD}}
\newcommand{\talpha}{\tilde{\alpha}}
\newcommand{\tnabla}{\tilde{\nabla}}


\textheight 9truein
\textwidth 6.5truein
\addtolength{\oddsidemargin}{-0.25in}
\addtolength{\evensidemargin}{-0.25in}
\addtolength{\topmargin}{-0.5in}
\setlength{\parindent}{0em}
\setlength{\parskip}{2ex}


\begin{document}
\maketitle


\section{Base cosmological FLD equation}
\label{sec:PDE}

The radiation energy density, flux, and pressure tensor are related
to the specific intensity moments via the relations
\begin{align}
  \label{eq:density_wrt_intensity}
  \Enu &= \frac{4\pi}{c} J_{\nu} = \frac1c \oint I_{\nu} \,d\Omega, \\
  \label{eq:flux_wrt_intensity}
  \Fnu^i &= 4 \pi H^i_{\nu} = \oint \vec{n}^i I_{\nu} \,d\Omega, \\
  \label{eq:pressure_wrt_intensity}
  \Pnu^{ij} &= \frac{4\pi}{c} K^{ij}_\nu = \frac1c \oint \vec{n}^i
    \vec{n}^j I_{\nu} \,d\Omega,
\end{align}
where each of these are defined in proper, CGS units.  
For frequencies $\nu\in\R^+$, times $t\in\R$ and spatial locations
$\rvec\in\R^3$, we denote the domain of these functions as $\Omegabar
= \R^+\times\R\times\R^3$. Then $I_{\nu}:\Omegabar\rightarrow\R$ is
the specific radiation intensity,$\Enu:\Omegabar\rightarrow\R$ is the
radiation energy density, $\Fnu:\Omegabar\rightarrow\Rthree$ is the
radiation energy flux, and $\Pnu:\Omegabar\rightarrow\R^{3\times 3}$
is the radiation pressure tensor. 

The moment equations for a generalized fluid coupled with a radiation
fluid for a cosmological medium are as follows
(c.f.~\cite{HayesNorman2003,Paschos2005}).  The zeroth moment of the 
Boltzmann equation provides an evolution equation for the radiation
energy density,
\begin{equation}
\label{eq:zero_mom}
  \partial_{t} E_{\nu} + \nabla\cdot(\Enu\vb) 
    + \nabla\cdot\Fnu + \Pnu:(\nabla\vb) + \frac{3\adot}{a}\Enu -
    \frac{\nu\adot}{a}\partial_{\nu} \Enu
  = \eta_{\nu} -c \kappa_{\nu} \Enu.
\end{equation}
Here, $\eta_{\nu}:\Omegabar\rightarrow\R$ is the emissivity source
(g cm$^{-1}$ s$^{-2}$).  $\kappa_{\nu}:\Omegabar\rightarrow\R$ is the
combined opacity (cm$^{-1}$) due to the elemental species, and is
computed as 
\begin{equation}
\label{eq:opacity_def}
  \kappa_{\nu} = \sum_{i=1}^{N_\text{chem}} \sigma_{i}(\nu)\,\mn_i,
\end{equation}
where $\sigma_i(\nu)$ is the cross-section of the elemental species
$\mn_i$.  We note that in the above equation (and the remainder of
this document), the spatial derivatives denoted by $\nabla$ are taken
with respect to the proper position, $\rvec$.

Similarly, the first moment of the Boltzmann equation provides an
evolution equation for the radiation energy flux,
\begin{equation}
\label{eq:first_mom}
  \partial_{t} \Fnu + \nabla\(\Fnu\cdot\vb\) 
    + c^2 \nabla\cdot\Pnu + \(\Fnu\cdot\nabla\)\vb
    + \frac{3 \adot}{a} \Fnu - \frac{\nu \adot}{a}\partial_{\nu} \Fnu
  = -c \kappa_{\nu} \Fnu.
\end{equation}



\subsection{Flux Limited Diffusion approximation}
\label{subsec:fld_approx}

For a flux limited diffusion (FLD) approximation, the radiative flux
vector is computed as a function of the energy density gradient through
a parameterization, 
\begin{equation}
\label{eq:fld_approx}
  \Fnu = -D\,\nabla\Enu,
\end{equation}
where $D:\Omegabar\rightarrow\R^{3\times3}$ is the {\em flux limiter}
(cm$^2$ s$^{-1}$) that depends on both the opacity $\kappa_\nu$, and
the radiation energy density $\Enu$.

Using the FLD approximation, the equation \eqref{eq:first_mom} may be
ignored, and the zeroth moment equation \eqref{eq:zero_mom} becomes
\begin{equation}
\label{eq:mgfld}
\begin{split}
   \partial_{t} \Enu &+ \nabla\cdot(\Enu\vb) 
     - \nabla\cdot(D\,\nabla\Enu)
     - \frac{1}{c}\(\nabla(D\,\nabla\Enu)\):(\nabla\vb) 
     + \frac{3 \adot}{a} \Enu - \frac{\nu \adot}{a}\partial_{\nu}\Enu \\
   &= \eta_{\nu} - c \kappa_{\nu} \Enu,
\end{split}
\end{equation}
where we have used the FLD-based approximation of the radiation
pressure as $\Pnu = \frac{1}{c}\nabla\Fnu =
\frac{1}{c}\nabla\(D\,\nabla\Enu\)$.  Note that in general, $\Pnu$ is  
non-symmetric, due to the spatial dependence of the flux-limiter $D$.
The equation \eqref{eq:mgfld} is an equation defining the
scalar-valued variable $\Enu$, that may either be solved
independently, or coupled with the elemental densities $\mn_i$ and
hydrodynamic quantities of the the fluid energy $e$, velocity
$\vb$ and density $\rhob$.

For conditions that allow omission of the radiation pressure term, the
zeroth moment equations corresponding to \eqref{eq:zero_mom} can be
simplified to
\begin{align}
  \label{eq:mgfld_simplified}
  &\partial_{t} \Enu + \nabla\cdot(\Enu\vb) 
    - \nabla\cdot(D\,\nabla\Enu) 
     + \frac{3 \adot}{a} \Enu - \frac{\nu \adot}{a}\partial_{\nu}\Enu
    = \eta_{\nu} - c \kappa_{\nu} \Enu.
\end{align}
The equation \eqref{eq:mgfld_simplified} is a
reaction-advection-diffusion equation, that is again a function of the
energy density $\Enu$, though here is parabolic in nature, in that it
describes a diffusion-like radiation transport.  

Lastly, since Enzo performs a second-order-accurate piecewise
parabolic algorithm for hydrodynamic advection, with passive advection
of other density-like quantities (including $\Enu$), in considering a
new solver for evolving only the radiation field, we may consider the
simpler reaction-diffusion equation
\begin{align}
  \label{eq:mgfld_simplified2}
  &\partial_{t} \Enu - \nabla\cdot(D_{\nu}\,\nabla\Enu) 
      + \frac{3 \adot}{a} \Enu - \frac{\nu \adot}{a}\partial_{\nu}\Enu
    = \eta_{\nu} - c \kappa_{\nu} \Enu.
\end{align}




\subsection{Single Group Approximation}
\label{subsec:single_group}

The radiation energy density equation \eqref{eq:mgfld_simplified2} may
be discretized in frequency space by integrating the energy density
(or the flux) between two frequencies, namely $\nu_{g+1}$ and
$\nu_{g}$.  This forms one radiation energy 'group', which provides an
average estimate of the moment quantity as:
\[
   E_{g}(t,\rvec) = \frac{1}{\nu_{g+1}-\nu_{g}} \int_{\nu_{g}}^{\nu_{g+1}}
   \Enu(\nu,t,\rvec)\,\mathrm d\nu, \quad g=1,\ldots,N_g.
\]
We take this approximation further, through going to the limit of a
single grey radiation energy density $E$ (g cm$^{-1}$ s$^{-2}$), by
integrating in frequency space from the ionization threshold of
hydrogen, $\nu_{HI}$, to infinity.  We accomplish this by assuming
the existence of a given radiation frequency spectrum, $\chi_E(\nu)$,
such that the frequency-dependent radiation energy density may be
written in the form $\Enu(\nu,t,\rvec)=\tilde{E}(t,\rvec)\,\chi_E(\nu)$.  
We note that this is a rather strong assumption, since it requires
that all radiation throughout the space-time domain has the same
frequency spectrum.  With this, we define a single ``grey'' radiation
energy density, 
\begin{equation}
\label{eq:grey_definition}
  E(t,\rvec) = \int_{\nu_0}^{\infty} \Enu(\nu,t,\rvec) \mathrm d\nu =
  \tilde{E}(t,\rvec) \int_{\nu_0}^{\infty} \chi_{E}(\nu)\mathrm d\nu.
\end{equation}
We note that this approximation is valid only when the assumed
frequency spectrum is integrable, i.e. if it is defined only over a
limited frequency bandwidth, or if it scales with frequency as
$E_{\nu} \propto \nu^{-\beta_{q}}$ for some $\beta_{q} > 1$.  The
latter case holds for quasar and stellar type spectra. 

Application of such an integration in equation
\eqref{eq:mgfld_simplified2} yields the following ``grey'' 
radiation diffusion equation:
\begin{align}
  \label{eq:mgfld_grey}
  \partial_{t} E - \nabla\cdot\(D\,\nabla E\)
    + \alpha\frac{\adot}{a} E = \eta - c\kappa E.
\end{align}
In the above equation, the term 
$-\frac{\nu \adot}{a}\partial_{\nu}\Enu$ is replaced (through
integration by parts) with a contribution to the cosmology expansion
term by an amount equal to $\frac{\adot}{a} E$.  However, if the
assumed spectrum $\chi_E(\nu)$ is monochromatic, i.e. $\chi_E(\nu) =
\delta_{\nu_0}(\nu)$, then integration by parts does not yield this
contribution.  Hence, we use the parameter $\alpha$ to account for
this difference, in that for monochromatic radiation spectra we use
$\alpha=3$, but for all other spectra we use $\alpha=4$.

We note that through using the integrated radiation energy density
$E$, since it is based on an {\em a-priori} defined function
$\chi_E(\nu)$, we may compute all frequency-space integrals based on
$\Enu$ at initialization of the radiation module, and store the
integrated weights for reuse throughout the simulation.  This pertains
to the frequency-integrated opacity $\kappa$, in which 
\begin{align}
  \notag
  & c\,\kappa\,E \ = \ c \int_{\nu_0}^{\infty} \kappa_{\nu}\,\Enu\,\mathrm d\nu, \\
  \Longrightarrow\quad &\notag \\
  \label{eq:opacity_integrated}
  &\kappa \ = \ \frac{\int_{\nu_0}^{\infty} \kappa_{\nu}\,\Enu\,\mathrm d\nu}{
  \int_{\nu_0}^{\infty} \Enu\,\mathrm d\nu} \ = \ 
  \frac{\sum_{i=1}^{N_\text{chem}} \mn_i \int_{\nu_i}^{\infty}
    \chi_E\,\sigma_{\mn_i}\,\mathrm d\nu}{
  \int_{\nu_0}^{\infty} \chi_E\,\mathrm d\nu}.
\end{align}
Similarly, we may compute the integrated photoionization rate
$\Gamma_i^{ph}$ as
\[
   \Gamma_i^{ph} 
   \ = \ 
   c \int_{\nu_i}^{\infty} \frac{\sigma_{\mn_i}\Enu}{h\nu}\,\mathrm d\nu
   \ = \ 
   \frac{cE \int_{\nu_i}^{\infty}
     \frac{\chi_E\,\sigma_{\mn_i}}{\nu}\,\mathrm d\nu}{
   h \int_{\nu_0}^{\infty} \chi_E\,\mathrm d\nu}.
\]
We note that the above two quantities, and in fact all quantities
involving the frequency integrated radiation energy density, may be
written in terms of 
\begin{align}
  \label{eq:integrals} 
    \int_{\nu_0}^{\infty} \chi_E\,\mathrm d\nu,  \qquad
    \int_{\nu_i}^{\infty} \chi_E\,\sigma_{\mn_i}\,\mathrm d\nu,  \quad\text{and}\quad
    \int_{\nu_i}^{\infty} \frac{\chi_E\,\sigma_{\mn_i}}{\nu}\,\mathrm d\nu.
\end{align}
Since these integrals depend only on the {\em a-priori} defined
functions $\chi_E$ and $\sigma_{\mn_i}$, we may compute these once at
initialization and reuse them throughout any grey FLD-based
simulation.  To this end, we consider the integral change of variables
defined through introduction of $\omega = \nu^{-1}$:
\begin{equation}
\label{eq:integral_transform}
\begin{split}
  \int_{\nu_0}^{\infty} f({\nu})\, \mathrm d{\nu} &\ = \ 
      \nu_0 \int_0^1 \frac{1}{\omega^2}\,f\left(\frac{{\nu}_0}{\omega}\right)\,\mathrm d\omega \ = \ 
      \nu_0\sum_{i=1}^s \int_{\omega_{i-1}}^{\omega_i} \frac{1}{\omega^2}\,
      f\left(\frac{{\nu}_0}{\omega}\right)\,\mathrm d\omega \\
    &\ \approx \ \nu_0\sum_{i=1}^s \frac{\omega_i-\omega_{i-1}}{6} \left[
      \frac{1}{\omega_{i-1}^2}\,f\left(\frac{{\nu}_0}{\omega_{i-1}}\right) +
      \frac{1}{(\omega_{i-1}+\omega_i)^2}\,f\left(\frac{2{\nu}_0}{\omega_{i-1}+\omega_i}\right) +
      \frac{1}{\omega_i^2}\,f\left(\frac{{\nu}_0}{\omega_i}\right) \right],
\end{split}
\end{equation}
where we have partitioned the interval $(0,1)$ into $s$
subintervals, $\omega_0<\omega_1<\ldots<\omega_s$.  Since the integrand does not
exist at either of the points 0 or 1, we choose $\omega_0=\varepsilon$ and
$\omega_s = 1-\varepsilon$, pick a regular spacing $\omega_i-\omega_{i-1}=1/s$, and
choose $s=10000$ in order to reduce the error in the numerical
integration to $\mathcal O(\varepsilon + 1/s^4)$. 

Moreover, in the case of a grey radiation energy field, we may
consider a ``grey'' emissivity through integrating $\eta_{\nu}(\nu,t,\rvec)$ 
over $\nu\in(\nu_0,\infty)$ to determine its single-group
approximation
\begin{equation}
\label{eq:eta_T}
  \eta(t,\rvec) =\int_{\nu_0}^{\infty} \eta_{\nu}(\nu,t,\rvec)\,\mathrm d\nu.
\end{equation}
We note that this approach implicitly assumes that the spectrum of
$\eta_{\nu}$ matches that of $\Enu$.




\subsection{Flux Limiter}

In the grey flux-limited diffusion approximation
\eqref{eq:mgfld_grey}, the role of the flux limiter
$D$ is one of providing a continuous transition between the isotropic
and free-streaming limits.  To this end, we consider the flux limiter
to be of the form,
\[
   D(\kappa,E) = \left[\begin{array}{ccc} 
       D_1(\kappa,E) & 0 & 0 \\
       0 & D_2(\kappa,E) & 0 \\
       0 & 0 & D_3(\kappa,E) 
     \end{array}\right],
\]
where we employ a limiter of the form  \cite{Morel2000}: 
\begin{align}
  \label{eq:Morel_limiterD}
  D_i(\kappa,E) &= \min\left\{ c \left(9\kappa_f^2 + R_i^2\right)^{-1/2},
    D_\text{max} \right\}, \quad i=1,2,3,\\
  \label{eq:Morel_limiterR}
  R_i &= \max\left\{ \frac{|\partial_{\rvec_i}E|}{E_f}, R_\text{min} \right\}.
\end{align}
Here, the spatial derivative within $R$ is computed at the
computational face adjoining two neighboring finite-volume cells, as
will be elaborated on further in section \ref{sec:fv_approximation}.
We enforce bounds on $D_i$ and $R_i$ to ensure numerical stability due
to floating-point arithmetic error, given by 
%$D_\text{max} = 0.006\,c\,\Lunit$ 
$D_\text{max} = \infty$ 
and 
$R_\text{min} = 10^{-20} \Lunit^{-1}$, 
with $\Lunit$ the length non-dimensionalization factor in the simulation
(see the next section).  Moreover, the face-centered radiation energy
density and opacity are computed using the arithmetic and harmonic
means, respectively,
\[
  E_f = \frac{E_1 + E_2}{2}, \qquad \kappa_f =
  \frac{2\kappa_1\kappa_2}{\kappa_1+\kappa_2}, 
\]
where here $E_1$ and $E_2$ are the two values of $E$ in the cells
adjacent to the face, with $\kappa_1$ and $\kappa_2$ defined
similarly. 




\section{Units in Enzo cosmology}
\label{sec:units}

When run with cosmological expansion enabled, Enzo modifies the
units of each internal field throughout the simulation to account for
cosmological expansion.  We define the comoving position $\xvec$
through the relationship $\rvec = \Lunit\xvec \propto a \xvec$
(specific details are below in equation \eqref{eq:Lunit}), where $a$
is the cosmological expansion factor.  This is a time-dependent
(equivalently, redshift-dependent) factor, satisfying the relationship
\begin{align}
  \label{eq:CurrentRedshift}
  z &= \frac{1}{a} - 1,
\end{align}
where $z$ is the current (time-dependent) redshift.  We note that
since $z$ decreases as time proceeds, $a$ increases as a function of
time, since 
\begin{align}
  \label{eq:expansion_factor}
  a = \frac{1}{1+z}.
\end{align}

Enzo then defines the non-dimensionalization factors:
\begin{align}
  \label{eq:Dunit}
  \Dunit &= (1.88\times10^{-29}) \Omega_{mn} H_{cn}^2 (1 + z)^3, \\
  \label{eq:Lunit}
  \Lunit &= \frac{(3.086\times10^{24}) L_c}{H_{cn} (1 + z)}
          \qquad \left(\text{i.e.}\quad \Lunit = \frac{(3.086\times10^{24}) L_c}{H_{cn}}a\right), \\
  \label{eq:Tunit}
  \Tunit &= \frac{2.52\times10^{17}}{\Omega_{mn}^{1/2} H_{cn} (1 + z_I)^{3/2}}, \\
  \label{eq:Vunit}
  \Vunit &= (1.225\times10^{7}) L_c \sqrt{\Omega_{mn}(1 + z_I)}, \\
  \label{eq:Aunit}
  \Aunit &= (1+z_I)^{-1}.
\end{align}
Here $a = \Aunit \tA$, where $\tA$ is Enzo's normalized value,  
$\Omega_{mn}$ is Enzo's {\tt OmegaMatterNow} parameter, 
$H_{cn}$ is Enzo's {\tt HubbleConstantNow} parameter, 
$z_I$ is Enzo's {\tt InitialRedshift} parameter, and
$L_c$ is Enzo's {\tt ComovingBoxSize} parameter.
As a result, Enzo actually computes the current redshift via the
formula 
\begin{align}
  \label{eq:CurrentRedshiftFormula}
  z &= \frac{1 + z_I}{\tA} - 1.
\end{align}
We note that under these definitions, $\Vunit \ne \Lunit / \Tunit$, since
\begin{align*}
   \frac{\Lunit}{\Tunit} &= 
   \frac{(3.086\times10^{24}) L_c \left(\Omega_{mn}^{1/2} H_{cn} (1 + z_I)^{3/2}\right)}
         {H_{cn} (1 + z)\left(2.52\times10^{17}\right)} \\
   &= 
   (1.2246\times10^{7}) L_c \sqrt{\Omega_{mn}(1 + z_I)} \frac{(1 + z_I)}{(1 + z)},
\end{align*}
which differs from $\Vunit$ both in the precision of the leading
constant (minor difference) and also by a factor of $(1+z_I)/(1+z)$, that grows
in magnitude as time proceeds, especially for simulations having large
initial redshift (major difference).

With these units defined, we may also consider a unit normalization
factor for radiation energy density, that can be defined as either
\[
  \Dunit \Vunit^2, \quad\text{or}\quad 
  \Dunit \frac{\Lunit^2}{\Tunit^2},
\]
since both constitute the correct CGS units.  However, since in
cosmological simulations $\Vunit \ne \Lunit / \Tunit$, these two
definitions are not the same.  We choose the first of these,
\begin{equation}
  \label{eq:Eunit}
  \Eunit = \Dunit \Vunit^2.
\end{equation}




\section{Recasting to comoving, normalized form}
\label{sec:comoving_eqn}

Our equations \eqref{eq:mgfld_grey} and
\eqref{eq:Morel_limiterD}-\eqref{eq:Morel_limiterR}  
are valid in proper CGS units, whereas Enzo's fields are stored in
comoving, normalized form.  Specifically, the terms in our equations
relate to Enzo's comoving, normalized terms in the following manner:
\begin{itemize}
\item $\rhob = \Dunit \tRho$, where $\rhob$ is the proper, CGS
  density, and $\tRho$ is Enzo's comoving, normalized
  density value,
\item $\vb = \Vunit \tilde{\vb}$, where $\vb$ is the proper peculiar
  baryonic CGS velocity, and $\tilde{\vb}$ is Enzo's comoving,
  normalized velocity value,
\item $e = \Vunit^2 \tilde{e}$, where $e$ is the proper
  baryonic energy per unit mass, and $\tilde{e}$ is Enzo's comoving,
  normalized energy value,
\item $E = \Eunit \tE$, where $E$ is the proper, CGS radiation
  energy, and $\tE$ is Enzo's comoving, normalized
  radiation energy value,
\item $a = \Aunit \tA$, as described above in equation
  \eqref{eq:Aunit},
\item $\adot = \left(\Aunit/\Tunit\right) \dot{\tA}$, where $\adot$ is
  the true cosmological expansion rate, and $\dot{\tA}$ is Enzo's
  normalized value,
\item $\rvec = \Lunit \xvec$, where $\rvec$ is the
  proper, CGS position, and $\xvec$ is Enzo's comoving normalized position,
\item $t = \Tunit \tT$, where $t$ is the CGS time, and $\tT$ is Enzo's
  normalized time,
\item $\kappa = \tK \Kunit$, where $\kappa$ is the proper, CGS
  opacity, $\tK$ is Enzo's normalized value, and $\Kunit \propto
  \Lunit^{-1}$.
\end{itemize}

We also note that in order for us to convert our equations
for the proper CGS radiation energy ($E$) to the normalized,
comoving radiation energy density ($\tE$) we must consider how our
time derivative must be modified.  Specifically, since $E = \Eunit
\tE$, then by the product rule,
\[
   \partial_t E = \partial_t \left(\Eunit \tE\right) = 
   \Eunit\, \partial_t \tE + \tE\, \partial_t \Eunit.
\]
Due to our choice of $\Eunit = \Dunit \Vunit^2 \propto (1+z)^3$, we then have 
\[
   \partial_t \Eunit = \frac{3\, \Eunit\, \partial_t z}{1+z}.
\]
Moreover, because $z = \frac{1}{a}-1$ we have $\partial_t z =
-\frac{\adot}{a^2}$, and since $a = \frac{1}{1+z}$, this becomes 
\[
   \partial_t \Eunit = -3\Eunit\frac{\adot}{a}.
\]
As a result, we have 
\begin{align*}
   \partial_t E &= \Eunit\, \partial_t \tE - 3\Eunit \frac{\adot}{a} \tE \\
   &= \Eunit\, \partial_t \tE - 3\frac{\adot}{a} E,
\end{align*}
which will cancel a portion of the term $\alpha\frac{\adot}{a} E$ in
the equation \eqref{eq:mgfld_grey}.  Specifically, after dividing
through by $\Eunit$ and canceling like terms, the equation
\eqref{eq:mgfld_grey} may instead be written in terms of the comoving,
normalized radiation energy density: 
\begin{align}
  \label{eq:mgfld_grey_Ecomoving}
  \partial_{t} \tE - \nabla\cdot\(D\,\nabla \tE\)
    + \talpha\frac{\adot}{a} \tE = \frac{\eta}{\Eunit} - c\kappa \tE,
\end{align}
where now for monochromatic radiation spectra $\talpha=0$ and
otherwise $\talpha=1$.

Similarly, since the proper position changes as a function of time,
then we may consider spatial differentiation with respect to the
normalized comoving position $\xvec$.  Since 
\[
   \rvec = \Lunit \xvec \quad\Longleftrightarrow\quad
   \xvec = \frac{\rvec}{\Lunit},
\]
then the chain rule dictates that
\[
   \frac{\partial}{\partial \rvec} \ = \
   \frac{\partial \xvec}{\partial \rvec}
   \frac{\partial}{\partial \xvec} \ = \
   \frac{1}{\Lunit}\frac{\partial}{\partial \xvec}.
\]
We therefore denote the comoving, normalized differentiation operator
as $\tnabla$, such that 
\[
   \nabla = \frac{1}{\Lunit}\tnabla.
\]
Resultingly, upon converting to comoving, normalized units, we have
the equation
\begin{align}
  \label{eq:mgfld_grey_comoving}
  \partial_{t} \tE - \frac{1}{\Lunit^2}\tnabla\cdot\(D\,\tnabla \tE\)
    + \talpha\frac{\adot}{a} \tE = \frac{\eta}{\Eunit} - c\kappa \tE.
\end{align}

Similarly, we may consider time differentiation with respect to our
normalized time value, $\tT$:
\[
   \frac{\partial}{\partial t} \ = \
   \frac{\partial \tT}{\partial t} \frac{\partial}{\partial \tT} \ = \
   \frac{1}{\Tunit}\frac{\partial}{\partial \tT},
\]
resulting in the equation
\begin{align}
  \label{eq:mgfld_grey_comoving_tnormalized}
  \partial_{\tT} \tE - \frac{\Tunit}{\Lunit^2}\tnabla\cdot\(D\,\tnabla \tE\)
    + \talpha\frac{\Tunit \adot}{a} \tE = \frac{\Tunit\eta}{\Eunit} -
    \Tunit c\kappa \tE.
\end{align}

Lastly, we will convert the equation \eqref{eq:mgfld_grey_comoving} so
that it depends only on Enzo's normalized variables, $\tT$, $\tE$,
$\tA$, $\dot{\tA}$ and $\tK$.  To this end, we expand each
variable in terms of its normalized value and unit factor, 
\begin{align}
  \notag
  &\partial_{\tT} \tE - \frac{\Tunit}{\Lunit^2}\tnabla\cdot\(D\,\tnabla \tE\)
    + \talpha\frac{\Tunit \Aunit\tAdot}{\Tunit \Aunit\tA} \tE = 
    \frac{\Tunit\eta}{\Eunit} - \Tunit \Kunit c \tK \tE \\
  \Leftrightarrow \quad & \\
  \label{eq:mgfld_grey_comoving_normalized}
  &\partial_{\tT} \tE - \frac{\Tunit}{\Lunit^2}\tnabla\cdot\(D\,\tnabla \tE\)
    + \talpha\frac{\tAdot}{\tA} \tE = 
    \frac{\Tunit\eta}{\Eunit} - \Tunit \Kunit\, c\, \tK \tE,
\end{align}
where we compute the limiter as
\begin{align}
  \label{eq:Morel_limiterD_normalized}
  D_i(\tK,\tE) &= \min\left\{ c \left(9\kappa_f^2 + R_i^2\right)^{-1/2},
    D_\text{max} \right\}, \quad i=1,2,3,\\
  \label{eq:Morel_limiterR_normalized}
  R_i &= \max\left\{ \frac{|\partial_{\xvec_i}\tE|}{\Lunit \tE_f}, R_\text{min} \right\},
\end{align}
with
\begin{align}
  \label{eq:face_values_normalized}
  \tE_f = \frac{\tE_1 + \tE_2}{2}, \qquad 
  \kappa_f = \frac{2\tK_1\tK_2}{\tK_1+\tK_2}\Kunit,
\end{align}
where again $\tE_1$ and $\tE_2$ are the two values of $\tE$ in the
cells adjacent to the face, as elaborated on in the next section. 






\section{Finite volume PDE approximation}
\label{sec:fv_approximation}

On a uniform (i.e.~non-AMR) mesh with comoving grid spacings $\Delta x$,
$\Delta y$ and $\Delta z$, we define the finite volume cell centers at
the grid points in comoving, normalized units as:
\begin{align*}
   \xvec_{i,j,k} &= \left[\,x_i,\,  y_j,\, z_k\,\right], \\
   x_i &= \left(i + \tfrac12\right) \Delta x, \\
   y_j &= \left(j + \tfrac12\right) \Delta y, \\
   z_k &= \left(k + \tfrac12\right) \Delta z.
\end{align*}
Enzo's data arrays contain the discrete values of each field over the
simulation volume at each of these grid points.  More specifically,
Enzo uses three-dimensional data arrays to store these discretized
solution values at specific points in time.  We denote $\tE^n_{i,j,k}$
as our approximation to the comoving and normalized solution at the
normalized time $\tT_{n}$ and at the comoving normalized spatial
location $\xvec_{i,j,k}$ (and do similarly for the other field data).
We must therefore consider how to discretize the space and time
derivatives of our equation \eqref{eq:mgfld_grey_comoving_normalized}
so that it depends on only these discrete data values.

We first separate the space and time discretizations.  First, we
discretize in time using a one-step $\theta$-method.  Introducing the
notation
\[
   \mD(\tE,\tK,\eta,\tA,\tAdot) = 
   \frac{\Tunit}{\Lunit^2}\tnabla\cdot\(D\,\tnabla \tE\)
    - \talpha\frac{\tAdot}{\tA} \tE 
    + \frac{\Tunit\eta}{\Eunit} 
    - \Tunit \Kunit\, c\, \tK \tE,
\]
and denoting $\tE^n$ as our approximation to the spatially continuous
solution at time $\tT_{n}$, the $\theta$-method may be written as
\begin{align}
  \label{eq:mgfld_theta}
  \tE^{n+1} - \tE^n &= 
    \theta\Delta \tT \mD(\tE^{n+1},\tK^{n+1},\eta^{n+1},\tA^{n+1},\tAdot^{n+1}) 
    + (1-\theta)\Delta \tT \mD(\tE^n,\tK^n,\eta^n,\tA^n,\tAdot^n),
\end{align}
where in the first $\mD$ term we lag the implicit dependence of the
solution on the flux limiter $D^{n+1}$ to the previous time, $D^n$ to
result in a linearly implicit system of equations.  

We must similarly apply our finite-volume spatial discretization of
the operator $\mD$.  For the discretized equation centered at the
comoving normalized position $\xvec_{i,j,k}$, we have 
\begin{align}
  \label{eq:mgfld_discrete}
  \tE_{i,j,k}^{n+1} - \tE_{i,j,k}^n &= \theta\Delta \tT \mD_{i,j,k}^{n+1} 
    + (1-\theta)\Delta \tT \mD_{i,j,k}^{n},
\end{align}
where
\begin{align}
  \label{eq:discrete_operator}
  \mD_{i,j,k} &= 
       \frac{\Tunit}{\Lunit^2 \Delta x^2}\(D_{i+1/2,j,k}\(\tE_{i+1,j,k} - \tE_{i,j,k}\) - D_{i-1/2,j,k}\(\tE_{i,j,k} - \tE_{i-1,j,k}\)\) \\
 \notag
    &+ \frac{\Tunit}{\Lunit^2 \Delta y^2}\(D_{i,j+1/2,k}\(\tE_{i,j+1,k} - \tE_{i,j,k}\) - D_{i,j-1/2,k}\(\tE_{i,j,k} - \tE_{i,j-1,k}\)\) \\
  \notag
    &+ \frac{\Tunit}{\Lunit^2 \Delta z^2}\(D_{i,j,k+1/2}\(\tE_{i,j,k+1} - \tE_{i,j,k}\) - D_{i,j,k-1/2}\(\tE_{i,j,k} - \tE_{i,j,k-1}\)\) \\
  \notag
    &- \talpha\frac{\tAdot}{\tA} \tE_{i,j,k} + \frac{\Tunit \eta_{i,j,k}}{\Eunit} 
     - \Tunit \Kunit\, c\, \tK_{i,j,k} \tE_{i,j,k},
\end{align}
and where, following equations
\eqref{eq:Morel_limiterD_normalized}-\eqref{eq:face_values_normalized},
we similarly discretize the limiter as, e.g.
\begin{align}
  \label{eq:discrete_limiter}
  D_{i+1/2,j,k}(\tE,\kappa) &= \min\left\{ c \left(9\kappa_{i+1/2,j,k}^2 + R_{i+1/2,j,k}^2\right)^{-1/2},
    D_\text{max} \right\},\\
  \notag
  R_{i+1/2,j,k} &= \max\left\{\frac{2}{\Lunit \Delta x} \frac{|\tE_{i+1,j,k}-\tE_{i,j,k}|}{\tE_{i+1,j,k}+\tE_{i,j,k}}, R_\text{min} \right\}, \\
  \notag
  \kappa_{i+1/2,j,k} &= \frac{2 \tK_{i+1,j,k} \tK_{i,j,k}}{\tK_{i+1,j,k} + \tK_{i,j,k}}\Kunit.
\end{align}
We note that in many of the above factors, the unit values change
as a function of time.  As a result, all quantities are converted
using the unit evaluated at the appropriate time step, $t_{n+1}$ or
$t_n$.



\subsection{Implemented equation}
\label{sec:fv_implementation}

Although the equations above represent what should be a
self-consistent implementation of the finite-volume discretized
version of our model, they are not exactly the same as those
implemented in Enzo.  I could not find the notes from my discussions
with Pascal on trying to get this model to work for our cosmological
I-front expansion tests, so I'm not sure how we got from the above
equations to the ones that follow.  Perhaps the derivation above
starts from equations with incorrect assumptions, or perhaps the
modifications we made just got lucky in reproducing those test
problems.  I will note, however, that those tests never began with a
large initial redshift, as we always tested $z_I = 4$ or $z_I = 10$.
As a result, the discrepancy between $\Vunit$ and $\Lunit/\Tunit$ may
be involved.

In any case, here are our implemented formulas akin to equations 
\eqref{eq:mgfld_discrete}-\eqref{eq:discrete_limiter}.  Terms that
differ from the above model are highlighted in {\color{red} red}:
\begin{align}
  \label{eq:mgfld_implemented}
    \tE_{i,j,k}^{n+1} - \tE_{i,j,k}^n &= \theta \Delta \tT
    \tmD_{i,j,k}^{n+1} + (1-\theta) \Delta \tT \tmD_{i,j,k}^n, \\
  \notag
  \tmD_{i,j,k} &= 
       \frac{\Tunit {\color{red}\tA^2}}{\Lunit^2 \Delta x^2}\(D_{i+1/2,j,k}\(\tE_{i+1,j,k} - \tE_{i,j,k}\) - D_{i-1/2,j,k}\(\tE_{i,j,k} - \tE_{i-1,j,k}\)\) \\
  \notag
    &+ \frac{\Tunit {\color{red}\tA^2}}{\Lunit^2 \Delta y^2}\(D_{i,j+1/2,k}\(\tE_{i,j+1,k} - \tE_{i,j,k}\) - D_{i,j-1/2,k}\(\tE_{i,j,k} - \tE_{i,j-1,k}\)\) \\
  \notag
    &+ \frac{\Tunit {\color{red}\tA^2}}{\Lunit^2 \Delta z^2}\(D_{i,j,k+1/2}\(\tE_{i,j,k+1} - \tE_{i,j,k}\) - D_{i,j,k-1/2}\(\tE_{i,j,k} - \tE_{i,j,k-1}\)\) \\
  \notag
    &- \talpha\frac{\dot{\tA}}{\tA} \tE_{i,j,k} + \frac{\Tunit \eta_{i,j,k}}{\Eunit}
     - \Tunit \Kunit\, c\, \tK_{i,j,k} \tE_{i,j,k},
\end{align}
where the face-centered quantities are computed from the normalized
values as
\begin{align}
  \label{eq:limiter_implemented}
  D_{i+1/2,j,k}(\tE,\kappa) &= \min\left\{ c \left(9\kappa_{i+1/2,j,k}^2 + R_{i+1/2,j,k}^2\right)^{-1/2},
    D_\text{max} \right\},\\
  \notag
  R_{i+1/2,j,k} &= \max\left\{\frac{2{\color{red}\tA}}{\Lunit \Delta x} \frac{|\tE_{i+1,j,k}-\tE_{i,j,k}|}{\tE_{i+1,j,k}+\tE_{i,j,k}}, R_\text{min} \right\}, \\
  \notag
  \kappa_{i+1/2,j,k} &= \frac{2 \tK_{i+1,j,k} \tK_{i,j,k}}{
    \tK_{i+1,j,k} + \tK_{i,j,k}} \Kunit.
\end{align}





\bibliography{sources}
\bibliographystyle{siam}
\end{document}
